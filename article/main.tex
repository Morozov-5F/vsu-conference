%!BIB program = biber

%%%%%%%%%%%%%%%%%%%%%%%%%%%%%%%%%%%%%%%%%
% University Assignment Title Page 
% LaTeX Template
% Version 1.0 (27/12/12)
%
% This template has been downloaded from:
% http://www.LaTeXTemplates.com
%
% Original author:
% WikiBooks (http://en.wikibooks.org/wiki/LaTeX/Title_Creation)
%
% License:
% CC BY-NC-SA 3.0 (http://creativecommons.org/licenses/by-nc-sa/3.0/)
% 
% Instructions for using this template:
% This title page is capable of being compiled as is. This is not useful for 
% including it in another document. To do this, you have two options: 
%
% 1) Copy/paste everything between \begin{document} and \end{document} 
% starting at \begin{titlepage} and paste this into another LaTeX file where you 
% want your title page.
% OR
% 2) Remove everything outside the \begin{titlepage} and \end{titlepage} and 
% move this file to the same directory as the LaTeX file you wish to add it to. 
% Then add \input{./title_page_1.tex} to your LaTeX file where you want your
% title page.
%
%%%%%%%%%%%%%%%%%%%%%%%%%%%%%%%%%%%%%%%%%
%\title{Title page with logo}
%----------------------------------------------------------------------------------------
%   PACKAGES AND OTHER DOCUMENT CONFIGURATIONS
%----------------------------------------------------------------------------------------

\documentclass[14pt]{extarticle}
\usepackage[english, russian]{babel}
\usepackage[utf8]{inputenc}
\usepackage[T2A]{fontenc}
\usepackage{amsmath, amsthm, amsfonts}
\usepackage{graphicx}
\usepackage[colorinlistoftodos]{todonotes}
\usepackage{geometry}
\numberwithin{equation}{section}
\usepackage[nottoc]{tocbibind}
%\usepackage{mathptmx}
%\usepackage{fontspec}
%\setmainfont{Times New Roman}

\usepackage[style=gost-numeric,
	language=auto,
	autolang=other,
	sorting=nty,
	backend=biber
]{biblatex}
\addbibresource{../bib_files/bibliography.bib}

\newtheorem{proposal}{Предложение}
\newtheorem{theorem}{Теорема}
\newtheorem{lemma}{Лемма}
\newtheorem{theorem_alt}{Теорема (альтернативная формулировка)}
\newtheorem{consequence}{Следствие}

\begin{document}

\begin{titlepage}
\newgeometry{margin=2cm}
\newcommand{\HRule}{\rule{\linewidth}{0.5mm}} % Defines a new command for the horizontal lines, change thickness here

\center % Center everything on the page
 
%----------------------------------------------------------------------------------------
%   HEADING SECTIONS
%----------------------------------------------------------------------------------------

\textsc{МИНОБРНАУКИ РОССИИ ФЕДЕРАЛЬНОЕ ГОСУДАРСТВЕННОЕ БЮДЖЕТНОЕ ОБРАЗОВАТЕЛЬНОЕ
УЧРЕЖДЕНИЕ ВЫСШЕГО ОБРАЗОВАНИЯ <<ВОРОНЕЖСКИЙ ГОСУДАРСТВЕННЫЙ УНИВЕРСИТЕТ>>
}\\[1.5cm] % Name of your university/college
\textsc{Факультет компьютерных наук}\\ % Major heading such as course name
\textsc{Кафедра цифровых технологий}\\[1.0cm]

%----------------------------------------------------------------------------------------
%   TITLE SECTION
%----------------------------------------------------------------------------------------

\HRule \\[0.4cm]
{ \huge \bfseries Системы билинейных уравнений}\\[0.4cm] % Title of your document
\HRule \\[1.5cm]

\textsc{Курсовая работа}\\[0.2cm]
\textit{02.03.01 Математика и компьютерные науки}\\
\textit{\small(010200 Математика и компьютерные науки)}\\[0.2cm]
\textit{Распределенные системы и искусственный интеллект}\\[4.0cm]

%----------------------------------------------------------------------------------------
%   AUTHOR SECTION
%----------------------------------------------------------------------------------------

\begin{minipage}{0.18\textwidth}
\begin{flushleft}
\textrm{Зав.кафедрой:}\\[0.2cm]
\textrm{Обучающийся:}\\[0.2cm]
\textrm{Руководитель:}\\
\end{flushleft}
\end{minipage}
~\begin{minipage}{0.7\textwidth}
\begin{flushright}
~\hrulefill~д. ф-м н. профессор \textsc{С.Д. Кургалин} \\ [0.2cm]
~\hrulefill~студент 3 к. \textsc{Е.Ю. Морозов} \\ [0.2cm]
~\hrulefill~д. ф-м н. профессор \textsc{А.В. Лобода} \\ [0.2cm]
\end{flushright}
\end{minipage}
% If you don't want a supervisor, uncomment the two lines below and remove the section above
%\Large \emph{Author:}\\
%John \textsc{Smith}\\[3cm] % Your name

%----------------------------------------------------------------------------------------
%   DATE SECTION
%----------------------------------------------------------------------------------------

%{\large \today}\\[2cm] % Date, change the \today to a set date if you want to be precise

%----------------------------------------------------------------------------------------
%   LOGO SECTION
%----------------------------------------------------------------------------------------

% \includegraphics{logo.png}\\[1cm] % Include a department/university logo - this will 
require the graphicx package
 
%----------------------------------------------------------------------------------------

\vfill % Fill the rest of the page with whitespace
\textsc{Воронеж}, 2016
\end{titlepage}

\newgeometry{margin=2cm}
\null\vspace{\fill}

%----------------------------------------------------------------------------------------
%   ВСТУПЛЕНИЕ 
%----------------------------------------------------------------------------------------
\begin{abstract}
Системы уравнений "--- одна из самых часто встречающихся конструкций в математике. Наиболее 
исследованы системы линейных уравнений "--- они возникают практически во всех областях, так 
или иначе связанных с математикой. Но чаще всего линейность достигается некоторыми 
<<трюками>>, например, пренебрежением множителями высокого порядка. В чисто математических 
задачах такое поведение не всегда допустимо, и, в связи с этим, возникает необходимость 
работы с системами нелинейных уравнений. К примеру, в задаче описания аффинно-однородных 
поверхностей в пространстве $ \mathbb{C}^3 $ возникают системы билинейных уравнений. В 
данной работе рассматривается именно этот тип систем и подробно будет разобран случай 
системы, состоящей из трёх уравнений относительно четырёх неизвестных.

\end{abstract}
\vspace{\fill}
\newpage
\tableofcontents
\newpage

%----------------------------------------------------------------------------------------
%   ПОСТАНОВКА ЗАДАЧИ
%----------------------------------------------------------------------------------------
\section{Постановка задачи}
Система билинейных уравнений (или билинейная система) "--- это система следующего вида: 
\begin{equation}\label{eq:bilinear_system}
	\sum_{j=1}^{m}\sum_{k=1}^{n} a^{i}_{j k} r_{j} s_{k} = d_i,~ i = \overline{1,l}, 
\end{equation}
где $m,n$ "--- число неизвестных в векторах $\mathbf{r}$ и $\mathbf{s}$, $l$ "--- число 
уравнений и $a^i_{jk} \in \mathbb{R}$ "--- действительные числовые коэффициенты~
\cite{stanford}.  Параметры системы могут быть коротко записаны в виде $ (l; m, n) $. В 
наиболее общем виде система 
записывается так: 
$$
	\mathbf{r}T\mathbf{s} = \mathbf{d},
$$
где $T = [a^{i}_{j k}] \in \mathbb{R}^{l \times m \times n}$ "--- тензор третьего ранга, 
составленный из коэффициентов системы (\ref{eq:bilinear_system}), $\mathbf{r}=[r_1,\cdots 
r_m]^T$ и $\mathbf{s}=[s_1,\cdots s_n]^T$ "--- векторы неизвестных, называемые 
\textit{наборами переменных}, $\mathbf{d}=[d_1,\cdots d_l]^T$ "--- вектор правой части. 
Предлагается также альтернативная запись системы 
(\ref{eq:bilinear_system}):
\begin{equation}\label{eq:forms}
	\mathbf{r}^T A_i \mathbf{s} = d_i,~ i = \overline{1,l}, 
\end{equation}
где $A_i$ "--- матрица билинейной формы следующего вида:
$$
	A_i = 
	\begin{pmatrix}
	a^{i}_{1 1} & \cdots & a^{i}_{1 n} \\
	\vdots      & \ddots & \vdots      \\
	a^{i}_{m 1} & \cdots & a^{i}_{m n} 
	\end{pmatrix}\in\mathbb{R}^{m\times n},~\text{для} ~ i = 1..\thinspace l
$$

Каждой билинейной системе можно сопоставить \textit{обобщенную матрицу системы}, 
составленную из развернутых в строки матриц билинейных форм каждого уравнения:
$$
	A = 
	\begin{pmatrix}
	a^{1}_{11} & \hdots  & a^{1}_{1n}  & \hdots &  a^{1}_{mn} \\ 
	\vdots &   & \ddots &   & \vdots \\
	a^{l}_{11} & \hdots  & a^{l}_{1n}  & \hdots  & a^{l}_{mn} \\ 
	\end{pmatrix} \in \mathbb{R}^{l \times m \cdot n}
$$ 
Ранг системы билинейных форм (\ref{eq:bilinear_system}) совпадает с рангом этой матрицы. В 
этой работе будут рассматриваться только системы полного ранга. 

Система называется однородной, если вектор правой части $\mathbf{d} = 0$: 
$$ 
	\sum_{j=1}^{m}\sum_{k=1}^{n} a_{i,j,k} r_{j} s_{k} = 0,~\text{для} ~ i = 1..\thinspace 
	l, 
$$

Простейшим примером билинейной системы может послужить уравнение равнобочной гиперболы в 
прямоугольной системе координат: 
\begin{equation}\label{eq:hyperbola}
	xy = \frac{a^2}{2},~ x\in\mathbb{R}, y\in\mathbb{R}, a\in\mathbb{R}
\end{equation}

Решением билинейной системы вида (\ref{eq:bilinear_system}) называются такие два вектора $
\mathbf{r}$ и $\mathbf{s}$, что при подстановке их в систему получаются верные равенства. 
Тривиальным называется решение, когда один из наборов переменных является нулевым. Вполне 
понятно, что любая однородная билинейная система всегда имеет тривиальные решения. 
Теории систем билинейных уравнений, которая позволяет описывать множества решений билинейных 
систем уравнений в общем виде, на данный момент не существует~\cite{stanford}. 
Тем не менее, в задаче об однородности интерес представляет наличие нетривиальных наборов 
переменных в таких системах. В данной работе как раз будет изучаться вопрос наличия 
нетривиальных решений у билинейной системы. Системы, имеющие нетривиальные решения, будем 
называть \textit{нерегулярными}. 

%----------------------------------------------------------------------------------------
%   ОБЩИЙ СЛУЧАЙ
%----------------------------------------------------------------------------------------
\newpage
\section{Общий случай} 

Как уже было сказано, для однородных билинейных систем не существует теории, которая бы 
позволила решать их в общем виде. Тем не менее, в некоторых математических задачах, к 
примеру, в задаче описания аффинно-однородных поверхностей в пространстве $ \mathbb{C}^3 $, 
возникает такая потребность. Один из возникающих сложных случаев связан с 
$$
	m = n,~l = 2m - 1.
$$
Особый интерес в уже упомянутой задаче представляют именно нерегулярные системы. В основном 
случае $(15; 8, 8)$ общая картина нетривиальных решений пока что не получена. В связи с 
этим, изучаются более простые случаи. Самый простой из них "--- $(1; 1, 1)$. Это одно 
билинейное уравнение вида
$$
	rs = 0
$$
Она является регулярной "--- у такой системы имеются только тривиальные решения. Далее 
будет рассмотрен менее тривиальный случай "--- $(3; 2, 2)$.

%----------------------------------------------------------------------------------------
%   ЧАСТНЫЙ СЛУЧАЙ
%----------------------------------------------------------------------------------------
\newpage
\section{Частный случай}
Рассмотрим следующую однородную симметричную билинейную систему: 
\begin{equation}\label{eq:trivial}
	\begin{cases}
		a_{1,1} r_1 s_1 + a_{1,2} r_1 s_2 + a_{1,3} r_2 s_1 + a_{1,4} r_2 s_2 = 0 \\
		a_{2,1} r_1 s_1 + a_{2,2} r_1 s_2 + a_{2,3} r_2 s_1 + a_{2,4} r_2 s_2 = 0 \\
		a_{3,1} r_1 s_1 + a_{3,2} r_1 s_2 + a_{3,3} r_2 s_1 + a_{3,4} r_2 s_2 = 0 \\
	\end{cases}
\end{equation}
Предлагается исследовать наличие нетривиальных решений в такой системе.

\subsection{Достаточное условие регулярности}

Для начала, рассмотрим набор $\mathbf{r}$. Вынесем $r_1, r_2$ как общие множители во всех 
уравнениях:
\begin{equation}\label{eq:trivial_r}
	\begin{cases}
		r_1(a_{1,1} s_1 + a_{1,2} s_2) + r_2(a_{1,3} s_1 + a_{1,4} s_2) = 0 \\
		r_1(a_{2,1} s_1 + a_{2,2} s_2) + r_2(a_{2,3} s_1 + a_{2,4} s_2) = 0 \\
		r_1(a_{3,1} s_1 + a_{3,2} s_2) + r_2(a_{3,3} s_1 + a_{3,4} s_2) = 0 \\
	\end{cases}
\end{equation}
Этой системе соответствует следующее матричное уравнение:
$$
	\underbrace{
	\begin{pmatrix}
		a_{1,1} s_1 + a_{1,2} s_2 & a_{1,3} s_1 + a_{1,4} s_2 \\
		a_{2,1} s_1 + a_{2,2} s_2 & a_{2,3} s_1 + a_{2,4} s_2 \\
		a_{3,1} s_1 + a_{3,2} s_2 & a_{3,3} s_1 + a_{3,4} s_2 \\
	\end{pmatrix}}_{B}
	\times
	\begin{pmatrix}
		r_1 \\
		r_2
	\end{pmatrix}=
	\begin{pmatrix}
		0 \\
		0 \\
		0
	\end{pmatrix}
$$
Ясно, что при фиксированных значениях $\mathbf{s}$ эта система будет иметь нетривиальные по 
$\mathbf{r}$ решения, если ранг матрицы $B$ не полон (в противном случае система имеет 
единственное решение "--- $\mathbf{r} = \mathbf{0}$)~\cite{costrikin_va1}.
Это равносильно тому, что все миноры второго порядка в матрице $B$ 
обращаются в ноль. Вообще говоря, миноры будут квадратично зависеть от $\mathbf{s}$. Таким 
образом, можно составить нелинейную систему уравнений относительно $s_1$ и $s_2$: 

\begin{equation}\label{eq:non_linear}
	\begin{cases}
		M_1 = Q_1(\mathbf{s}) = 0 \\
		M_2 = Q_2(\mathbf{s}) = 0 \\
		M_3 = Q_3(\mathbf{s}) = 0 \\
	\end{cases},
\end{equation}
где $M_1, M_2, M_3$ "--- миноры 2-го порядка в матрице $A$ и имеют следующий вид:
$$
\begin{matrix}
\begin{aligned}
M_1 &= (a_{1,1}a_{2,3} - a_{1,3}a_{2,1}) {s_{1}}^{2} + (a_{1,1}a_{2,4} + a_{1,2}a_{2,3} - 
a_{1,3}a_{2,2} - a_{1,4}a_{2,1}) s_{2}s_{1} + \\ & + (a_{1,2}a_{2,4} - a_{1,4}a_{2,2}) 
{s_{2}}^{2} 
\end{aligned}\\
\begin{aligned}
M_2 &= (a_{1,1}a_{3,3} - a_{1,3}a_{3,1}) {s_{1}}^{2} + (a_{1,1}a_{3,4} + a_{1,2}a_{3,3} - 
a_{1,3}a_{3,2} - a_{1,4}a_{3,1}) s_{2}s_{1} + \\ & + (a_{1,2}a_{3,4} - a_{1,4}a_{3,2}) 
{s_{2}}^{2} 
\end{aligned}\\
\begin{aligned}
M_3 &= (a_{2,1}a_{3,3} - a_{2,3}a_{3,1}) {s_{1}}^{2} + (a_{2,1}a_{3,4} + a_{2,2}a_{3,3} -a 
_{2,3}a_{3,2} - a_{2,4}a_{3,1}) s_{2}s_{1} + \\ & +(a_{2,2}a_{3,4} - a_{2,4}a_{3,2}) 
{s_{2}}^{2} \end{aligned}
\end{matrix}
$$

Очевидно, что в системе (\ref{eq:non_linear}) тривиальный набор $\mathbf{s} = \{s_1, s_2\}$, уже 
рассмотренный ранее, является решением. В таком случае, чтобы выбрать нетривиальные решения в системе 
(\ref{eq:trivial_r}), необходимо найти нетривиальные решения системы (\ref{eq:non_linear}). Для этого 
перепишем систему в виде линейной относительно мономов второго порядка ($s_1 s_2, s_1^2, s_2^2$):
\begin{equation}\label{eq:linear_monoms}
	M \times \begin{pmatrix} s_1 s_2 \\ s_1 ^ 2 \\ s_2 ^ 2 \end{pmatrix} = \mathbf{0}
\end{equation}
Введём следующие обозначения: 
$$
	s_1 s_2 = Y_1, s_1^2 = Y_2, s_2^2 = Y_3
$$
Таким образом, система (\ref{eq:linear_monoms}) приобретает вид:
\begin{equation}\label{eq:linear_y}
	M \times \mathbf{Y} = \mathbf{0}
\end{equation}
Эта система является линейной по $\mathbf{Y}$, что позволяет рассматривать её решения с классической 
точки зрения линейной алгебры.

Легко видеть, что $\mathbf{Y} = \mathbf{0}$ является решением системы (\ref{eq:linear_y}) и не 
зависит от вида матрицы $M$. Если рассматривать $M$ как матрицу некоего линейного оператора 
$\mathcal{A} : \mathbb{R}^3 \to \mathbb{R}^3 $, то справедливо следующее отношение:
$$
	\textrm{dim}(\textrm{im}\,\mathcal{A}) + \textrm{dim}(\textrm{ker}\,\mathcal{A}) = \textrm{dim}\,
	\mathbb{R}^3 = 3
$$
Отсюда вытекает уравнение на количество решений в системе (\ref{eq:linear_y}):
$$
	\textrm{rank}\,M + \textrm{dim}\,\mathbf{Y} = 3
$$

Рассмотрим случай, когда ранг матрицы $M$ полон, т.е. $\textrm{rank}\,M = 3$. В этом случае $M$ 
является невырожденной матрицей. Сказанное выше приводит к $\textrm{dim}\,\mathbf{Y} = 0$, из чего 
следует, что $\mathbf{Y} = \mathbf{0}$. Для такой матрицы $M$, с учетом введённых обозначений, 
получается, что: 
$$
	\begin{cases}
		s_1 s_2 = 0 \\
		s_1 ^ 2 = 0 \\
		s_2 ^ 2 = 0
	\end{cases}
$$
Если квадратичные комбинации из компонентов $\mathbf{s}$ нулевые, то $s_1 = s_2 = 0$ "--- тривиальное решение.Отметим, что критерием невырожденности матрицы является её определитель. Таким образом, если определитель матрицы $M$ не равен нулю, то матрица не является вырожденной. 

В явном виде этот определитель имеет члены шестого порядка и выглядит довольно громоздко. Для того, чтобы сократить запись, введём новое обозначение: пусть 
$d_{i, j, k}$ "--- минор второго порядка в обобщенной матрице системы $A$, полученный вычеркиванием $i$-й строки и столбцов $j$ и $k$. Тогда определитель матрицы $M$ может быть записан в следующем (компактном) виде: 
$$
	\begin{aligned}
	\det M & = (d_{2,1,4} d_{3,2,4} + d_{2,2,3} d_{3,2,4} - d_{2,2,4} d_{3,1,4} - 
		 	d_{2,2,4} d_{3,2,3}) d_{1,1,3} + \\
		 	& + (d_{1,2,4} d_{3,1,4} + d_{1,2,4} d_{3,2,3} - d_{1,1,4} d_{3,2,4} - d_{1,2,3}
		 	d_{3,2,4}) d_{2,1,3} + \\		 
		 	& + (d_{1,1,4} d_{2,2,4} + d_{1,2,3}d_{2,2,4} - d_{1,2,4}d_{2,1,4} - d_{1,2,4}
		 	d_{2,2,3}) d_{3,1,3}
	\end{aligned}
$$

Следует отметить, что аналогичные рассуждения относительно набора $\mathbf{s}$ приводят к точно такому же определителю. 

Таким образом, из всего сказанного выше, можно сформулировать \textit{достаточное условие регулярности} системы:
\begin{theorem}\label{thm:regularity}
Если определитель матрицы $M$ не равен нулю, то исходная система билинейных уравнений регулярна:
$$
\det M \ne 0 \implies\text{исходная система регулярна} 
$$
\end{theorem}

Во многих задачах представляют интерес нерегулярные системы (т.е. системы, которые имеют 
нетривиальные решения). В таком случае, из теоремы \ref{thm:regularity} можно сформулировать следующее следствие, являющееся необходимым условием нерегулярности исходной системы:

\begin{consequence}
	Если система билинейных уравнений не регулярна, то определитель матрицы $M$ равен нулю:
	$$
	\text{исходная система не регулярна} \implies \det M = 0 
	$$
\end{consequence}

\noindent Приведем пример:
\begin{equation*}
	\begin{cases}
		r_1 s_1 = 0 \\
		r_2 s_2 = 0 \\
		6 r_1 s_2 + 8 r_2 s_1 = 0 \\
	\end{cases}
\end{equation*}
Для этой матрицы построим мономно-минорную матрицу $M$ и найдем её определитель:
$$
	M =
	\begin{pmatrix}
	0 & -1 & \;\;\,0 \\
	0 & \;\;\,0 & -6 \\
	8 & \;\;\,0 & \;\;\,0 
	\end{pmatrix}
	\implies 
	\det M = 48 \ne 0
$$
Таким образом, эта система является регулярной. Действительно, эта система имеет только тривиальные решения (это можно легко проверить с помощью простого перебора вариантов).

%----------------------------------------------------------------------------------------
%   ПРИВЕДЕНИЕ К КАНОНИЧЕСКОМУ ВИДУ
%----------------------------------------------------------------------------------------
\subsection{Приведение к каноническому виду}
Перепишем $(3; 2, 2)$-систему в альтернативном виде~(\ref{eq:forms}): 
\begin{equation}\label{eq:simple_forms}
	\begin{cases}
		\mathbf{r}^T \cdot A_1 \cdot \mathbf{s} = 0 \\ 
		\mathbf{r}^T \cdot A_2 \cdot \mathbf{s} = 0 \\ 
		\mathbf{r}^T \cdot A_3 \cdot \mathbf{s} = 0 \\ 
	\end{cases}, A_i \in \mathbb{R}^{2 \times 2}
\end{equation}
Обозначим \textit{допустимые преобразования}, которые не изменяют факта регулярности
произвольной $(l; m, n)$ -системы: 
\begin{itemize}
 	\item гауссовы преобразования уравнений в системе, 
	
	\item замены переменных $\mathbf{r} = C \cdot \mathbf{r}^*$, $\mathbf{s} = D \cdot 
	\mathbf{s}^*$, которые изменяют матрицу билинейной формы $A_i$ по закону:
	$$
		A^{*}_{i} = C^T \cdot A_i \cdot D,
	$$ где $C$, $D$ "--- невырожденные матрицы соответствующих размерностей.
\end{itemize}

Для дальнейших преобразований потребуется хотя бы одно уравнение с невырожденной матрицей 
билинейной формы. Если матрицы всех трех форм вырождены, то встает вопрос о том, можно ли 
выбрать такую линейную комбинацию билинейных форм, чтобы матрица итоговой формы была 
невырождена. 

\begin{lemma}
	Если $(3;2,2)$-система является системой полного ранга, то существует линейная 
	комбинация билинейных форм из этой системы, имеющая невырожденную матрицу.
\end{lemma}
\begin{proof}
	Рассмотрим обобщенную матрицу $(3;2,2)$-системы:
	$$
		A =  
	 		\begin{pmatrix}
				a_{11} & a_{12} & a_{13} & a_{14} \\
				a_{21} & a_{22} & a_{23} & a_{24} \\
				a_{31} & a_{32} & a_{33} & a_{34} \\
			\end{pmatrix}
	$$ 
	Выше было сказано, что ранг этой матрицы совпадает с рангом соответствующей системы, 
	поэтому из полноты ранга исходной системы следует полнота ранга обобщенной матрицы. 
	Следовательно, в матрице $A$  существует ненулевой минор третьего порядка. Пусть, не 
	теряя общности, этим минором будет минор $M_1$, составленный из первых трёх столбцов 
	матрицы $A$:
	$$ 
		M_1 = a_{11}a_{22}a_{33} + a_{21}a_{32}a_{13} + a_{12}a_{23}a_{31}-a_{13}a_{22}
		a_{31}-a_{32}a_{23}a_{11}-a_{21}a_{12}a_{33}
	$$
	
	В таком случае, матрица $A$ приводима гауссовыми преобразованиями к ступенчатому виду~
	\cite{costrikin_va1}:
	$$
		A \sim A^* = 
		\begin{pmatrix}
			a^*_{11} & a^*_{12} & a^*_{13} & a^*_{14} \\
		 	     0   & a^*_{21} & a^*_{22} & a^*_{23} \\
			     0   &   0      & a^*_{31} & a^*_{32} \\
		\end{pmatrix}
	$$
	Так как $M_1\ne 0$, матрица $A^*$ приводима элементарными преобразованиями к следующему 
	виду: 
	$$
		A^* \sim A' =
		\begin{pmatrix}
			     1 & 0 & 0 & a'_{11} \\
		 	     0 & 1 & 0 & a'_{21} \\
			     0 & 0 & 1 & a'_{31} \\
		\end{pmatrix} 
	$$ 
	Этой новой обобщенной матрице системы соответствуют следующие билинейные формы: 
	$$
		A'_1 = 
		\begin{pmatrix}
			1 & 0 \\
			0 & a'_{11}\\
		\end{pmatrix},
		A'_2 = 
		\begin{pmatrix}
			0 & 1 \\
			0 & a'_{21}\\
		\end{pmatrix},
		A'_3 = 
		\begin{pmatrix}
			0 & 0 \\
			1 & a'_{31}\\
		\end{pmatrix}
	$$
	Сложим матрицы билинейных форм $A'_2$ и $A'_3$ и вычислим определитель этой комбинации: 
	$$
		B = A'_2 + A'_3,~\det B = -1,
	$$
	то есть матрица $B$ невырождена.
	Билинейная форма $B$ получена линейной комбинацией форм $A'_2$ и $A'_3$, а те, в свою 
	очередь, являются линейными комбинациями первоначальных билинейных форм (они были 
	получены преобразованиями гаусса исходной системы). Повторяя аналогичные рассуждения для 
	других миноров матрицы $A$, получается такой же результат. Лемма доказана.
\end{proof}

Используя эту лемму, можно получить систему с одной невырожденной матрицей 
билинейной формы. Пусть этой матрицей будет матрица $A_1$. В таком случае, <<улучшим>> вид 
матрицы $A_1$, введя замену вида: 
$$
	\mathbf{r} = C\mathbf{r}^*, \mathbf{s} = D\mathbf{s}^*, C = A_1^T, D = E
$$
Тогда матрица $A_1$ будет выглядеть следующим образом: 
$$ 
	A_1^* = (A_1^{-T})^T \cdot A_1 \cdot E = E
$$
Произведём еще одну замену так, чтобы <<улучшить>> вид формы $A_2$, сохраняя тривиальный 
(единичный) вид формы $A_1$:
\begin{align*}
	A_1^* & = C^T \cdot A_1 \cdot D = C^T \cdot E \cdot D \Rightarrow D = C^T \\
	A_2^* & = C^T \cdot A_2 \cdot C^{-T} = N^{-1} \cdot A_2 \cdot N,~\text{где}~N=C^{-T}
\end{align*}
Конструкция $N^{-1} \cdot A_2 \cdot N$ известна из курса линейной алгебры "--- оператор 
приведения матрицы линейного оператора к диагональному виду~\cite{costrikin_va2}. Таким образом, вид матрицы $A_2^*$ будет зависеть от собственных значений и собственных векторов матрицы $A_2$. Для матрицы $2 \times 2$ возможны следующие три случая~\cite{tyrtyshnikov_ma}: 
\begin{itemize}
	\item Различные собственные значения (простой спектр): $\lambda_1 \ne \lambda_2$;
	\item Кратные собственные значения: $\lambda_1 = \lambda_2$
	\item Комплексные собственные значения: $\lambda_{1,2} = \alpha \pm \imath \beta,~\beta 
	\ne 0$.
\end{itemize}
Рассмотрим каждый случай подробно. 

%----------------------------------------------------------------------------------------
%   ПРОСТОЙ СПЕКТР
%----------------------------------------------------------------------------------------
\subsubsection{Случай простого спектра}
Как известно из линейной алегбры, матрица, имеющая простой спектр диагонализируема и будет 
иметь на главной диагонали собственные значения~\cite{costrikin_va2}. В этом случае, билинейные формы системы после обозначенных замен приобретают следующий вид: 
\begin{equation}\label{eq:simple_raw}
	A^*_1 = 
	\begin{pmatrix}
		1 & 0 \\
		0 & 1 \\
	\end{pmatrix}, 
	A^*_2 = 
	\begin{pmatrix}
		\lambda_1 & 0 \\
		0 & \lambda_2
 	\end{pmatrix},
	A^*_3 = 
	\begin{pmatrix}
		a^*_{31} & a^*_{32} \\
		a^*_{33} & a^*_{34}
	\end{pmatrix},
\end{equation} 
где $a^*_{3 j} \in \mathbb{R}$ "--- коэффициенты третьей формы, полученные после замен.

Приведём формы~(\ref{eq:simple_raw}) элементарными преобразованиями к виду: 
\begin{equation}
	A'_1 = 
	\begin{pmatrix}
		0 & 0 \\
		0 & 1 \\
	\end{pmatrix}, 
	A'_2 = 
	\begin{pmatrix}
		1 & 0 \\
		0 & 0
 	\end{pmatrix},
	A'_3 = 
	\begin{pmatrix}
		0 & \cos \varphi \\
		\sin \varphi & 0
	\end{pmatrix},
\end{equation}
где $\varphi$ "--- параметр, полученный после линейных комбинаций уравнений.
Указанным билинейным формам будет соответствовать следующая система: 
\begin{equation}
	\begin{cases}
		r^*_2 s^*_2 = 0 \\ 
		r^*_1 s^*_1 = 0 \\
		r^*_1 s^*_2 \cos \varphi + r^*_2 s^*_1 \sin \varphi = 0
	\end{cases}
\end{equation} 
Нетривиальные решения в этой системе будут либо при $\sin \varphi = 0$, либо при $\cos 
\varphi = 0$. Таким образом, получаются первые два \textit{канонические} виды системы:
\begin{equation}
1)
	\begin{cases} 
		r^*_2 s^*_2 = 0 \\ 
		r^*_1 s^*_1 = 0 \\
		r^*_1 s^*_2 = 0
	\end{cases},
2)
	\begin{cases} 
		r^*_2 s^*_2 = 0 \\ 
		r^*_1 s^*_1 = 0 \\
		r^*_2 s^*_1 = 0
	\end{cases}
\end{equation}

%----------------------------------------------------------------------------------------
%   КРАТНЫЕ СОБСТВЕННЫЕ ЗНАЧЕНИЯ
%----------------------------------------------------------------------------------------
\subsubsection{Случай кратных собственных значений}
При кратных собственных значениях, матрица не является диагонализируемой, но является 
приводимой к Жордановой нормальной форме, известной из курса линейной алгебры
\cite{costrikin_va2}. Жорданова нормальная форма "--- это блочно-диагональная матрица с 
блоками вида 
$$
J_\lambda=
\begin{pmatrix}
	\lambda & 1       & 0             & \cdots & 0       & 0      \\
	0           & \lambda & 1             & \cdots & 0       & 0      \\
	0           & 0       & \lambda       & \ddots & 0       & 0      \\
	\vdots   & \vdots  & \ddots     & \ddots & \ddots  & \vdots \\
	0           & 0       & 0             & \ddots & \lambda & 1      \\
	0           & 0       & 0             & \cdots & 0       & \lambda \\
\end{pmatrix}
$$
Каждый такой блок $J_{\lambda}$ называется жордановой клеткой с собственным значением $
\lambda$. 

В таком случае, билинейные формы системы примут следующий вид:
\begin{equation}\label{eq:equal_raw}
	A^*_1 = 
	\begin{pmatrix}
		1 & 0 \\
		0 & 1 \\
	\end{pmatrix}, 
	A^*_2 = 
	\begin{pmatrix}
		\lambda & 1 \\
		0 & \lambda
 	\end{pmatrix},
	A^*_3 = 
	\begin{pmatrix}
		a^*_{31} & a^*_{32} \\
		a^*_{33} & a^*_{34}
	\end{pmatrix}
\end{equation}
С помощью линейных комбинаций этих форм, мы можем получить следующий упрощенный вид: 
\begin{equation}
	A'_1 = 
	\begin{pmatrix}
		1 & 0 \\
		0 & 1 \\
	\end{pmatrix}, 
	A'_2 = 
	\begin{pmatrix}
		0 & 1 \\
		0 & 0
 	\end{pmatrix},
	A'_3 = 
	\begin{pmatrix}
		a'_{31} & 0 \\
		a'_{33} & 0
	\end{pmatrix},
\end{equation}
$a'_{31},~a'_{33}$ одновременно не равны нулю (это следует из полноты ранга начальной 
системы).
Соответствующая этому упрощенному виду система: 
\begin{equation}
	\begin{cases}
		r^*_1s^*_1 + r^*_2s^*_2 = 0 \\
		r^*_1s^*_2 = 0 \\
		a'_{31}r^*_1s^*_1 + a'_{33}r^*_2s^*_1 = 0
	\end{cases}
\end{equation} 
Используя необходимое условие нерегулярности, приравняем к нулю определитель соответствующей матрицы $M$: 
$$
	\det M = {a'_{33}}^2 = 0 
$$
Это возможно при $a'_{33} = 0$. Легко видеть, что для новой системы существует нетривиальный 
набор решений $\{r^*_1 = 0, r^*_2 \ne 0, s^*_1 \ne 0, s^*_2 = 0\}$. Отсюда получается 
канонический вид, который совпадает с видом 1) из предыдущего пункта.  

%----------------------------------------------------------------------------------------
%   КОМПЛЕКСНЫЕ СОБСТВЕННЫЕ ЗНАЧЕНИЯ
%----------------------------------------------------------------------------------------
\subsubsection{Случай комплексных собственных значений}
В этом случае у матрицы $A_2$ имеются два собственных значения вида: 
$$
	\lambda = \alpha + \imath \beta,~\overline{\lambda} = \alpha - \imath \beta,~\beta \ne 0
$$ 
Как известно, матрицы с таким спектром приводятся к следующему виду \cite{tyrtyshnikov_ma}: 
$$ 
	A^*_2 = 
	\begin{pmatrix}
		\alpha & \beta \\
		-\beta & \alpha
	\end{pmatrix}
$$
В таком случае, билинейные формы после замен будут выглядеть следующим образом:
\begin{equation}
	A^*_1 = 
	\begin{pmatrix}
		1 & 0 \\
		0 & 1 \\
	\end{pmatrix}, 
	A^*_2 = 
	\begin{pmatrix}
		\alpha & \beta \\
		-\beta & \alpha
	\end{pmatrix},
	A^*_3 = 
	\begin{pmatrix}
		a^*_{31} & a^*_{32} \\
		a^*_{33} & a^*_{34}
	\end{pmatrix}
\end{equation}
Используя элементарные преобразования, получаем третий упрощенный вид системы: 
\begin{equation}\label{eq:complex_simplified}
	A^*_1 = 
	\begin{pmatrix}
		1 & 0 \\
		0 & 1 \\
	\end{pmatrix}, 
	A^*_2 = 
	\begin{pmatrix}
		0 & 1 \\
		-1 & 0
	\end{pmatrix},
	A^*_3 = 
	\begin{pmatrix}
		a'_{31} & 0 \\
		a'_{33} & 0
	\end{pmatrix}
\end{equation}
Здесь, как и в предыдущем пункте, $a'_{31},~a'_{33}$ одновременно не равны нулю. 
Соответствующая система будет иметь вид:
\begin{equation}
	\begin{cases}
		r^*_1s^*_1 + r^*_2s^*_2 = 0 \\ 
		r^*_1s^*_2 - r^*_2s^*_1 = 0 \\
		a'_{31}r^*_1s^*_1 + a'_{33}r^*_2s^*_1 = 0
	\end{cases}
\end{equation}
Воспользуемся необходимым условием нерегулярности: 
$$
	\det M = -({a'_{31}}^2 + {a'_{33}}^2)
$$
Он равен нулю только при $a'_{31} = a'_{33} = 0$, что недопустимо, поэтому упрощенный вид 
(\ref{eq:complex_simplified}) не может иметь нетривиальных решений. 

%----------------------------------------------------------------------------------------
%   КРИТЕРИЙ РЕГУЛЯРНОСТИ
%----------------------------------------------------------------------------------------
\subsubsection{Критерий регулярности для (3;2,2)-систем}
Рассмотрев детально каждый из случаев, можно получить критерий регулярности для произвольной 
$(3;2,2)$-системы полного ранга: 

\begin{theorem}
	Произвольная $(3;2,2)$-система полного ранга регулярна тогда и только тогда, когда она 
	не приводима элементарными преобразованиями к одному из двух канонических видов: 
	\begin{equation}
		1)
		\begin{cases} 	
			r_2 s_2 = 0 \\ 
			r_1 s_1 = 0 \\
			r_1 s_2 = 0
		\end{cases},
		2)
		\begin{cases} 
			r_2 s_2 = 0 \\ 
			r_1 s_1 = 0 \\
			r_2 s_1 = 0
		\end{cases}
	\end{equation}
\end{theorem}
Как следствие можно оформить критерий нерегулярности для произвольной $(3;2,2)$-системы: 
\begin{consequence}
	Произвольная $(3;2,2)$-система полного ранга нерегулярна тогда и только тогда, когда она 
	приводима элементарными преобразованиями к одному из двух канонических видов: 
	\begin{equation}
		1)
		\begin{cases} 	
			r_2 s_2 = 0 \\ 
			r_1 s_1 = 0 \\
			r_1 s_2 = 0
		\end{cases},
		2)
		\begin{cases} 
			r_2 s_2 = 0 \\ 
			r_1 s_1 = 0 \\
			r_2 s_1 = 0
		\end{cases}
	\end{equation}
\end{consequence}
Таким образом был получен критерий, позволяющий однозначно сказать о регулярности 
произвольной $(3;2,2)$ системы.

%----------------------------------------------------------------------------------------
%   ЗАКЛЮЧЕНИЕ
%----------------------------------------------------------------------------------------
\newpage
\addcontentsline{toc}{section}{Заключение}
\section*{Заключение}

В данной работе была рассмотрена проблема определения нерегулярности однородной системы 
билинейных уравнений. В общем случае эта задача слишком сложна для рассмотрения, поэтому 
были исследованы более простые случаи. Досконально был проанализирован случай $(3;2,2)$-
системы и был сформулирован критерий регулярности для таких систем. 

В данной работе не было сформулировано алгоритма приведения системы к каноническому виду, 
поэтому отдельно было представлено достаточное условие регулярности системы, которое можно 
использовать для исследования системы на регулярность. Метод, с помощью котором был получен 
данный критерий называется \textit{метод мономов и миноров}. Этот метод был использован при 
исследовании регулярности систем $(7; 4, 4)$ и $(8; 5, 5)$ и дал положительные результаты. В 
наиболее сложном случае $(15; 8, 8)$ возникает система линейных уравнений $6435 \times 
6435$, которую необходимо решить аналитически. С этой задачей не справился пока что ни один 
математический пакет.
%----------------------------------------------------------------------------------------
%   СПИСОК ЛИТЕРАТУРЫ
%----------------------------------------------------------------------------------------
\newpage
\addcontentsline{toc}{section}{Список литературы}
\printbibliography

\end{document}





