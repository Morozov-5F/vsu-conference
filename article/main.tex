%%%%%%%%%%%%%%%%%%%%%%%%%%%%%%%%%%%%%%%%%
% University Assignment Title Page 
% LaTeX Template
% Version 1.0 (27/12/12)
%
% This template has been downloaded from:
% http://www.LaTeXTemplates.com
%
% Original author:
% WikiBooks (http://en.wikibooks.org/wiki/LaTeX/Title_Creation)
%
% License:
% CC BY-NC-SA 3.0 (http://creativecommons.org/licenses/by-nc-sa/3.0/)
% 
% Instructions for using this template:
% This title page is capable of being compiled as is. This is not useful for 
% including it in another document. To do this, you have two options: 
%
% 1) Copy/paste everything between \begin{document} and \end{document} 
% starting at \begin{titlepage} and paste this into another LaTeX file where you 
% want your title page.
% OR
% 2) Remove everything outside the \begin{titlepage} and \end{titlepage} and 
% move this file to the same directory as the LaTeX file you wish to add it to. 
% Then add \input{./title_page_1.tex} to your LaTeX file where you want your
% title page.
%
%%%%%%%%%%%%%%%%%%%%%%%%%%%%%%%%%%%%%%%%%
%\title{Title page with logo}
%----------------------------------------------------------------------------------------
%   PACKAGES AND OTHER DOCUMENT CONFIGURATIONS
%----------------------------------------------------------------------------------------

\documentclass[12pt]{article}
\usepackage[english, russian]{babel}
\usepackage[utf8x]{inputenc}
\usepackage[T2A]{fontenc}
\usepackage{amsmath, amsthm, amsfonts}
\usepackage{graphicx}
\usepackage[colorinlistoftodos]{todonotes}
\usepackage{geometry}
\numberwithin{equation}{section}
\newtheorem{proposal}{Предложение}
\newtheorem{theorem}{Теорема}
\newtheorem{theorem_alt}{Теорема (альтернативная формулировка)}
\newtheorem{consequence}{Следствие}
\begin{document}

\begin{titlepage}

\newcommand{\HRule}{\rule{\linewidth}{0.5mm}} % Defines a new command for the horizontal lines, change thickness here

\center % Center everything on the page
 
%----------------------------------------------------------------------------------------
%   HEADING SECTIONS
%----------------------------------------------------------------------------------------

\textsc{\LARGE Воронежский государственный университет}\\[1.5cm] % Name of your university/college
\textsc{\Large Факультет компьютерных наук}\\[0.5cm] % Major heading such as course name

%----------------------------------------------------------------------------------------
%   TITLE SECTION
%----------------------------------------------------------------------------------------

\HRule \\[0.4cm]
{ \huge \bfseries Системы билинейных уравнений}\\[0.4cm] % Title of your document
\HRule \\[1.5cm]
 
%----------------------------------------------------------------------------------------
%   AUTHOR SECTION
%----------------------------------------------------------------------------------------

\begin{minipage}{0.4\textwidth}
\begin{flushleft} \large
\emph{Автор:}\\
Евгений \textsc{Морозов} % Your name
\end{flushleft}
\end{minipage}
~
\begin{minipage}{0.4\textwidth}
\begin{flushright} \large
\emph{Руководитель:} \\
Проф. А. В. \textsc{Лобода} % Supervisor's Name
\end{flushright}
\end{minipage}\\[2cm]

% If you don't want a supervisor, uncomment the two lines below and remove the section above
%\Large \emph{Author:}\\
%John \textsc{Smith}\\[3cm] % Your name

%----------------------------------------------------------------------------------------
%   DATE SECTION
%----------------------------------------------------------------------------------------

%{\large \today}\\[2cm] % Date, change the \today to a set date if you want to be precise

%----------------------------------------------------------------------------------------
%   LOGO SECTION
%----------------------------------------------------------------------------------------

% \includegraphics{logo.png}\\[1cm] % Include a department/university logo - this will require the graphicx package
 
%----------------------------------------------------------------------------------------

\vfill % Fill the rest of the page with whitespace
\textsc{Воронеж}, 2016
\end{titlepage}

\newgeometry{margin=2cm}
\null\vspace{\fill}
\begin{abstract}
Системы уравнений "--- одна из самых часто встречающихся конструкций в математике. Наиболее 
исследованы системы линейных уравнений "--- они возникают практически во всех областях, так или иначе 
связанных с математикой. Но чаще всего линейность достигается некоторыми <<трюками>>, например, 
пренебрежением множителями высокого порядка. В чисто математических задачах такое поведение не всегда 
допустимо, и, в связи с этим, возникает необходимость работы с системами нелинейных уравнений. К 
примеру, в задаче описания аффинно-однородных поверхностей в пространстве $ \mathbb{C}^3 $ возникают 
системы билинейных уравнений. В данной работе рассматривается именно этот тип систем и подробно будет 
разобран случай системы, состоящей из трёх уравнений относительно четырёх неизвестных.
\end{abstract}
\vspace{\fill}
\newpage
\tableofcontents
\newpage

\section{Определение системы билинейных уравнений}
Системой из $l$ билинейных уравнений называется система следующего вида: 
\begin{equation}\label{eq:bilinear_system}
	\sum_{j=1}^{m}\sum_{k=1}^{n} a_{i,j,k} r_{j} s_{k} = d_i,~\text{для} ~ i = 1..\thinspace l, 
\end{equation}
где $\mathbf{r} \in \mathbb{R}^m$, $\mathbf{s} \in \mathbb{R}^n$. Параметры системы могут быть коротко записаны в виде $ (l; m, n) $. В наиболее общем виде система 
записывается так: 
$$
	\mathbf{r}T\mathbf{s} = \mathbf{d},
$$
где $T = [a_{i,j,k}] \in \mathbb{R}^{l \times m \times n}$ "--- тензор третьего порядка, составленный 
из коэффициентов системы (\ref{eq:bilinear_system}), $\mathbf{r}=[r_1,\cdots r_m]^T$ и $\mathbf{s}
=[s_1,\cdots s_n]^T$ "--- векторы неизвестных, называемые \textit{наборами переменных}, $\mathbf{d}
=[d_1,\cdots d_l]^T$ "--- вектор правой части. Предлагается также альтернативная запись системы 
(\ref{eq:bilinear_system}):
\begin{equation}\label{eq:forms}
	\mathbf{r}^T A_i \mathbf{s} = d_i,~\text{для} ~ i = 1..\thinspace l, 
\end{equation}
где $A_i$ "--- матрица следующего вида:
$$
	A_i = 
	\begin{pmatrix}
	a_{i, 1, 1} & \cdots & a_{i, 1, n} \\
	\vdots      & \ddots & \vdots      \\
	a_{i, m, 1} & \cdots & a_{i, m, n} 
	\end{pmatrix}\in\mathbb{R}^{m\times n},~\text{для} ~ i = 1..\thinspace l
$$

Система называется однородной, если вектор правой части $\mathbf{d} = 0$: 
$$ 
	\sum_{j=1}^{m}\sum_{k=1}^{n} a_{i,j,k} r_{j} s_{k} = 0,~\text{для} ~ i = 1..\thinspace l, 
$$

Легко видеть, что при фиксировании одного из наборов переменных, получается обыкновенная система 
линейных уравнений относительно другого набора. Именно благодаря этому свойству билинейные системы 
получили своё название. 

Простейшим примером билинейной системы может послужить уравнение равнобочной гиперболы в 
прямоугольной системе координат: 
\begin{equation}\label{eq:hyperbola}
	xy = \frac{a^2}{2},~ x\in\mathbb{R}, y\in\mathbb{R}, a\in\mathbb{R}
\end{equation}

Если $m = n$, то такая система называется симметричной. При числе уравнений $l = 2n - 1$ симметричная 
система считается идеальной. Ясно, что уравнение (\ref{eq:hyperbola}) является примером идеальной системы 
билинейных уравнений. Такие системы представляют особый интерес в уже упомянутой задаче об описании 
аффинно-однородных поверхностей. 

Решением системы вида (\ref{eq:bilinear_system}) называются такие два вектора $\mathbf{r}$ и $
\mathbf{s}$, что при подстановке их в систему получаются верные равенства. Тривиальным называется 
решение, когда один из наборов переменных является нулевым. Соответственно, $\mathbf{r}$- или $
\mathbf{s}$-тривиальным решением называется такое решение, в котором соответствующий набор является 
нулевым. Отсюда следует, что любая однородная билинейная система всегда имеет тривиальные решения. 
Метода, позволяющего решить билинейную систему уравнений в общем виде на данный момент не существует. 
Тем не менее, в задаче об однородности интерес представляет наличие нетривиальных наборов переменных 
в таких системах. 

\section{Общий случай} 

В уже упомянтуой задаче описания аффинно-однородных поверхностей в пространстве $ \mathbb{C}^3 $ 
возникают однородные системы билинейных уравнений. Как показали исследования, большинство таких 
систем имеют только тривиальные решения. Для доказательства этого факта использовался 

%\textit{Сюда можно написать немного воды, рассказать про базис Грёбнера, редуцирование системы, и прочие 
%приятные вещи. Рассказать в общих чертах про метод МММ? Привести в качестве примера случай (15; 8, 8).}

\section{Частный случай}

Рассмотрим следующую однородную симметричную билинейную систему: 
\begin{equation}\label{eq:trivial}
	\begin{cases}
		a_{1,1} r_1 s_1 + a_{1,2} r_1 s_2 + a_{1,3} r_2 s_1 + a_{1,4} r_2 s_2 = 0 \\
		a_{2,1} r_1 s_1 + a_{2,2} r_1 s_2 + a_{2,3} r_2 s_1 + a_{2,4} r_2 s_2 = 0 \\
		a_{3,1} r_1 s_1 + a_{3,2} r_1 s_2 + a_{3,3} r_2 s_1 + a_{3,4} r_2 s_2 = 0 \\
	\end{cases}
\end{equation}
Предлагается исследовать наличие нетривиальных решений в такой системе.

\subsection{Достаточное условие регулярности}

Для начала, рассмотрим набор $\mathbf{r}$. Вынесем $r_1, r_2$ как общие множители во всех уравнениях:
\begin{equation}\label{eq:trivial_r}
	\begin{cases}
		r_1(a_{1,1} s_1 + a_{1,2} s_2) + r_2(a_{1,3} s_1 + a_{1,4} s_2) = 0 \\
		r_1(a_{2,1} s_1 + a_{2,2} s_2) + r_2(a_{2,3} s_1 + a_{2,4} s_2) = 0 \\
		r_1(a_{3,1} s_1 + a_{3,2} s_2) + r_2(a_{3,3} s_1 + a_{3,4} s_2) = 0 \\
	\end{cases}
\end{equation}
Этой системе соответствует следующее матричное уравнение:
$$
	\underbrace{
	\begin{pmatrix}
		a_{1,1} s_1 + a_{1,2} s_2 & a_{1,3} s_1 + a_{1,4} s_2 \\
		a_{2,1} s_1 + a_{2,2} s_2 & a_{2,3} s_1 + a_{2,4} s_2 \\
		a_{3,1} s_1 + a_{3,2} s_2 & a_{3,3} s_1 + a_{3,4} s_2 \\
	\end{pmatrix}}_{A}
	\times
	\begin{pmatrix}
		r_1 \\
		r_2
	\end{pmatrix}=
	\begin{pmatrix}
		0 \\
		0 \\
		0
	\end{pmatrix}
$$

Если $s_1=s_2=0$, то это, как уже было сказано, тривиальный набор, который обращает предыдущее уравнение 
в верное тождество. Аналогично с набором $r_1 = r_2 = 0$, который является единственным решением в 
системе выше, если ранг матрицы $A$ полон. 

Таким образом, если решение не является тривиальным, то существует такой нетривиальный набор $\mathbf{r}
\ne \{0, 0\}$ и такой набор $\mathbf{s} \ne \{0, 0\}$, что матрица $A$ имеет неполный ранг при 
соответствующем $\mathbf{s}$. Это равносильно тому, что все миноры второго порядка в матрице $A$ 
обращаются в ноль. Вообще говоря, миноры будут квадратично зависеть от $\mathbf{s}$. Таким образом, можно 
составить нелинейную систему уравнений относительно $s_1$ и $s_2$: 

\begin{equation}\label{eq:non_linear}
	\begin{cases}
		M_1 = Q_1(\mathbf{s}) = 0 \\
		M_2 = Q_2(\mathbf{s}) = 0 \\
		M_3 = Q_3(\mathbf{s}) = 0 \\
	\end{cases},
\end{equation}
где $M_1, M_2, M_3$ "--- миноры 2-го порядка в матрице $A$ и имеют следующий вид:
$$
\begin{matrix}
M_1 = (a_{1,1}a_{2,3} - a_{1,3}a_{2,1}) {s_{1}}^{2} + (a_{1,1}a_{2,4} + a_{1,2}a_{2,3} - a_{1,3}a_{2,2} - 
a_{1,4}a_{2,1}) s_{2}s_{1} + (a_{1,2}a_{2,4} - a_{1,4}a_{2,2}) {s_{2}}^{2} \\

M_2 = (a_{1,1}a_{3,3} - a_{1,3}a_{3,1}) {s_{1}}^{2} + (a_{1,1}a_{3,4} + a_{1,2}a_{3,3} - a_{1,3}a_{3,2} - 
a_{1,4}a_{3,1}) s_{2}s_{1} + (a_{1,2}a_{3,4} - a_{1,4}a_{3,2}) {s_{2}}^{2} \\

M_3 = (a_{2,1}a_{3,3} - a_{2,3}a_{3,1}) {s_{1}}^{2} + (a_{2,1}a_{3,4} + a_{2,2}a_{3,3} -a _{2,3}a_{3,2} - 
a_{2,4}a_{3,1}) s_{2}s_{1} + (a_{2,2}a_{3,4} - a_{2,4}a_{3,2}) {s_{2}}^{2} \\
\end{matrix}
$$

Очевидно, что в системе (\ref{eq:non_linear}) тривиальный набор $\mathbf{s} = \{s_1, s_2\}$, уже 
рассмотренный ранее, является решением. В таком случае, чтобы выбрать нетривиальные решения в системе 
(\ref{eq:trivial_r}), необходимо найти нетривиальные решения системы (\ref{eq:non_linear}). Для этого 
перепишем систему в виде линейной относительно мономов второго порядка ($s_1 s_2, s_1^2, s_2^2$):
\begin{equation}\label{eq:linear_monoms}
	M \times \begin{pmatrix} s_1 s_2 \\ s_1 ^ 2 \\ s_2 ^ 2 \end{pmatrix} = \mathbf{0}
\end{equation}
Введём следующие обозначения: 
$$
	s_1 s_2 = Y_1, s_1^2 = Y_2, s_2^2 = Y_3
$$
Таким образом, система (\ref{eq:linear_monoms}) приобретает вид:
\begin{equation}\label{eq:linear_y}
	M \times \mathbf{Y} = \mathbf{0}
\end{equation}
Эта система является линейной по $\mathbf{Y}$, что позволяет рассматривать её решения с классической 
точки зрения линейной алгебры.

Легко видеть, что $\mathbf{Y} = \mathbf{0}$ является решением системы (\ref{eq:linear_y}) и не 
зависит от вида матрицы $M$. Если рассматривать $M$ как матрицу некоего линейного оператора 
$\mathcal{A} : \mathbb{R}^3 \to \mathbb{R}^3 $, то справедливо следующее отношение:
$$
	\textrm{dim}(\textrm{im}\,\mathcal{A}) + \textrm{dim}(\textrm{ker}\,\mathcal{A}) = \textrm{dim}\,
	\mathbb{R}^3 = 3
$$
Отсюда вытекает уравнение на количество решений в системе (\ref{eq:linear_y}):
$$
	\textrm{rank}\,M + \textrm{dim}\,\mathbf{Y} = 3
$$

Рассмотрим случай, когда ранг матрицы $M$ полон, т.е. $\textrm{rank}\,M = 3$. В этом случае $M$ 
является невырожденной матрицей. Сказанное выше приводит к $\textrm{dim}\,\mathbf{Y} = 0$, из чего 
следует, что $\mathbf{Y} = \mathbf{0}$. Для такой матрицы $M$, с учетом введённых обозначений, 
получается, что: 
$$
	\begin{cases}
		s_1 s_2 = 0 \\
		s_1 ^ 2 = 0 \\
		s_2 ^ 2 = 0
	\end{cases}
$$
Если квадратичные комбинации из компонентов $\mathbf{s}$ нулевые, то $s_1 = s_2 = 0$ "--- тривиальное 
решение. Исходя из всего этого, предлагается следующая
\begin{theorem} \label{thm:theorem_one}
Если матрица $M$ невырождена, то система имеет только тривиальные решения.
\end{theorem}
Отметим, что критерием невырождености матрицы является её определитель. Таким образом, если 
определитель матрицы $M$ не равен нулю, то матрица не является вырожденной. В явном виде этот определитель выглядит довольно громоздко:

%Выпишем определитель матрицы $M$:
\begin{equation}\label{eq:determenant}
\begin{aligned}
\det{M} &= {a_{{1,1}}}^{2}a_{{2,2}}a_{{2,3}}{a_{{3,4}}}^{2} - {a_{{1,1}}}^{2}a_{{2,2}}a_{{2,4}}a_{{3,3}}a_{{3,4}} - {a_{{1,1}}}^{2}a_{{2,3}}a_{{2,4}}a_{{3,2}}a_{{3,4}} + {a_{{1,1}}}^{2}{a_{{2,4}}}^{2}a_{{3,2}}a_{{3,3}} -\\
        &-a_{{1,1}}a_{{1,2}}a_{{2,1}}a_{{2,3}}{a_{{3,4}}}^{2} + a_{{1,1}}a_{{1,2}}a_{{2,1}}a_{{2,4}}a_{{3,3}}a_{{3,4}} - a_{{1,1}}a_{{1,2}}a_{{2,2}}a_{{2,3}}a_{{3,3}}a_{{3,4}} + a_{{1,1}}a_{{1,2}}a_{{2,2}}a_{{2,4}}{a_{{3,3}}}^{2} +\\ 
        &+ a_{{1,1}}a_{{1,2}}{a_{{2,3}}}^{2}a_{{3,2}}a_{{3,4}} + a_{{1,1}}a_{{1,2}}a_{{2,3}}a_{{2,4}}a_{{3,1}}a_{{3,4}} - a_{{1,1}}a_{{1,2}}a_{{2,3}}a_{{2,4}}a_{{3,2}}a_{{3,3}} - a_{{1,1}}a_{{1,2}}{a_{{2,4}}}^{2}a_{{3,1}}a_{{3,3}} -\\ 
        &- a_{{1,1}}a_{{1,3}}a_{{2,1}}a_{{2,2}}{a_{{3,4}}}^{2} + a_{{1,1}}a_{{1,3}}a_{{2,1}}a_{{2,4}}a_{{3,2}}a_{{3,4}} + a_{{1,1}}a_{{1,3}}{a_{{2,2}}}^{2}a_{{3,3}}a_{{3,4}} - a_{{1,1}}a_{{1,3}}a_{{2,2}}a_{{2,3}}a_{{3,2}}a_{{3,4}} +\\
        &+ a_{{1,1}}a_{{1,3}}a_{{2,2}}a_{{2,4}}a_{{3,1}}a_{{3,4}} - a_{{1,1}}a_{{1,3}}a_{{2,2}}a_{{2,4}}a_{{3,2}}a_{{3,3}} + a_{{1,1}}a_{{1,3}}a_{{2,3}}a_{{2,4}}{a_{{3,2}}}^{2} - a_{{1,1}}a_{{1,3}}{a_{{2,4}}}^{2}a_{{3,1}}a_{{3,2}} +\\
        &+ a_{{1,1}}a_{{1,4}}a_{{2,1}}a_{{2,2}}a_{{3,3}}a_{{3,4}} + a_{{1,1}}a_{{1,4}}a_{{2,1}}a_{{2,3}}a_{{3,2}}a_{{3,4}} - 2\,a_{{1,1}}a_{{1,4}}a_{{2,1}}a_{{2,4}}a_{{3,2}}a_{{3,3}} - a_{{1,1}}a_{{1,4}}{a_{{2,2}}}^{2}{a_{{3,3}}}^{2} -\\
        &- 2\,a_{{1,1}}a_{{1,4}}a_{{2,2}}a_{{2,3}}a_{{3,1}}a_{{3,4}} + 2\,a_{{1,1}}a_{{1,4}}a_{{2,2}}a_{{2,3}}a_{{3,2}}a_{{3,3}} + a_{{1,1}}a_{{1,4}}a_{{2,2}}a_{{2,4}}a_{{3,1}}a_{{3,3}} - a_{{1,1}}a_{{1,4}}{a_{{2,3}}}^{2}{a_{{3,2}}}^{2} + \\ 
        &+ a_{{1,1}}a_{{1,4}}a_{{2,3}}a_{{2,4}}a_{{3,1}}a_{{3,2}} + {a_{{1,2}}}^{2}a_{{2,1}}a_{{2,3}}a_{{3,3}}a_{{3,4}} - {a_{{1,2}}}^{2}a_{{2,1}}a_{{2,4}}{a_{{3,3}}}^{2} - {a_{{1,2}}}^{2}{a_{{2,3}}}^{2}a_{{3,1}}a_{{3,4}} + \\ 
        &+ {a_{{1,2}}}^{2}a_{{2,3}}a_{{2,4}}a_{{3,1}}a_{{3,3}} + a_{{1,2}}a_{{1,3}}{a_{{2,1}}}^{2}{a_{{3,4}}}^{2} - a_{{1,2}}a_{{1,3}}a_{{2,1}}a_{{2,2}}a_{{3,3}}a_{{3,4}} - a_{{1,2}}a_{{1,3}}a_{{2,1}}a_{{2,3}}a_{{3,2}}a_{{3,4}} - \\ 
        &- 2\,a_{{1,2}}a_{{1,3}}a_{{2,1}}a_{{2,4}}a_{{3,1}}a_{{3,4}} + 2\,a_{{1,2}}a_{{1,3}}a_{{2,1}}a_{{2,4}}a_{{3,2}}a_{{3,3}} + 2\,a_{{1,2}}a_{{1,3}}a_{{2,2}}a_{{2,3}}a_{{3,1}}a_{{3,4}} - a_{{1,2}}a_{{1,3}}a_{{2,2}}a_{{2,4}}a_{{3,1}}a_{{3,3}} - \\
        &- a_{{1,2}}a_{{1,3}}a_{{2,3}}a_{{2,4}}a_{{3,1}}a_{{3,2}} + a_{{1,2}}a_{{1,3}}{a_{{2,4}}}^{2}{a_{{3,1}}}^{2} - a_{{1,2}}a_{{1,4}}{a_{{2,1}}}^{2}a_{{3,3}}a_{{3,4}} + a_{{1,2}}a_{{1,4}}a_{{2,1}}a_{{2,2}}{a_{{3,3}}}^{2} + \\ 
        &+ a_{{1,2}}a_{{1,4}}a_{{2,1}}a_{{2,3}}a_{{3,1}}a_{{3,4}} - a_{{1,2}}a_{{1,4}}a_{{2,1}}a_{{2,3}}a_{{3,2}}a_{{3,3}} + a_{{1,2}}a_{{1,4}}a_{{2,1}}a_{{2,4}}a_{{3,1}}a_{{3,3}} - a_{{1,2}}a_{{1,4}}a_{{2,2}}a_{{2,3}}a_{{3,1}}a_{{3,3}} + \\ 
        &+ a_{{1,2}}a_{{1,4}}{a_{{2,3}}}^{2}a_{{3,1}}a_{{3,2}} - a_{{1,2}}a_{{1,4}}a_{{2,3}}a_{{2,4}}{a_{{3,1}}}^{2} + {a_{{1,3}}}^{2}a_{{2,1}}a_{{2,2}}a_{{3,2}}a_{{3,4}} - {a_{{1,3}}}^{2}a_{{2,1}}a_{{2,4}}{a_{{3,2}}}^{2} -\\ 
        &- {a_{{1,3}}}^{2}{a_{{2,2}}}^{2}a_{{3,1}}a_{{3,4}} + {a_{{1,3}}}^{2}a_{{2,2}}a_{{2,4}}a_{{3,1}}a_{{3,2}} - a_{{1,3}}a_{{1,4}}{a_{{2,1}}}^{2}a_{{3,2}}a_{{3,4}} + a_{{1,3}}a_{{1,4}}a_{{2,1}}a_{{2,2}}a_{{3,1}}a_{{3,4}} - \\ 
        &- a_{{1,3}}a_{{1,4}}a_{{2,1}}a_{{2,2}}a_{{3,2}}a_{{3,3}} + a_{{1,3}}a_{{1,4}}a_{{2,1}}a_{{2,3}}{a_{{3,2}}}^{2} + a_{{1,3}}a_{{1,4}}a_{{2,1}}a_{{2,4}}a_{{3,1}}a_{{3,2}} + a_{{1,3}}a_{{1,4}}{a_{{2,2}}}^{2}a_{{3,1}}a_{{3,3}} - \\ 
        &- a_{{1,3}}a_{{1,4}}a_{{2,2}}a_{{2,3}}a_{{3,1}}a_{{3,2}} - a_{{1,3}}a_{{1,4}}a_{{2,2}}a_{{2,4}}{a_{{3,1}}}^{2} + {a_{{1,4}}}^{2}{a_{{2,1}}}^{2}a_{{3,2}}a_{{3,3}} - {a_{{1,4}}}^{2}a_{{2,1}}a_{{2,2}}a_{{3,1}}a_{{3,3}} - \\ 
        &- {a_{{1,4}}}^{2}a_{{2,1}}a_{{2,3}}a_{{3,1}}a_{{3,2}} + {a_{{1,4}}}^{2}a_{{2,2}}a_{{2,3}}{a_{{3,1}}}^{2}
\end{aligned}
\end{equation}

Для того, чтобы сократить запись этого выражения, составим новую матрицу, объединив развернутые в 
строки матрицы билинейных форм, соответствующие каждому уравнению, то есть $B = [a_{i, j}] \in 
\mathbb{R}^{3 \times 4}$. Введём новое обозначение: пусть $d_{k, j, k}$ "--- минор второго 
порядка в матрице $B$, полученный вычеркиванием $i$-й строки и столбцов $j$ и $k$. Тогда 
определитель матрицы $M$ может быть записан в следующем (компактном) виде: 
$$
	\begin{aligned}
	\det M & = (d_{2,1,4} d_{3,2,4} + d_{2,2,3} d_{3,2,4} - d_{2,2,4} d_{3,1,4} - 
		 	d_{2,2,4} d_{3,2,3}) d_{1,1,3} + \\
		 	& + (d_{1,2,4} d_{3,1,4} + d_{1,2,4} d_{3,2,3} - d_{1,1,4} d_{3,2,4} - d_{1,2,3}
		 	d_{3,2,4}) d_{2,1,3} + \\		 
		 	& + (d_{1,1,4} d_{2,2,4} + d_{1,2,3}d_{2,2,4} - d_{1,2,4}d_{2,1,4} - d_{1,2,4}
		 	d_{2,2,3}) d_{3,1,3}
	\end{aligned}
$$

Следует отметить, что аналогичные рассуждения относительно набора $\mathbf{s}$ приводят к точно 
такому же определителю. 

Введём новое понятие: система билинейных уравнений называется \textit{регулярной}, если она 
имеет только тривиальные решения. Таким образом, теорема \ref{thm:theorem_one} является 
\textbf{достаточным} условием регулярности системы. Поэтому, теорему \ref{thm:theorem_one} можно 
переформулировать в виде: 
\begin{theorem_alt}
Если определитель матрицы $M$ не равен нулю, то исходная система билинейных уравнений регулярна:
$$
\det M \ne 0 \implies\text{исходная система регулярна} 
$$
\end{theorem_alt}

Во многих задачах представляют интерес нерегулярные системы (т.е. системы, которые имеют 
нетривиальные 
решения). В таком случае, из теоремы \ref{thm:theorem_one} можно сформулировать следующее 
следствие, 
являющееся необходимым условием нерегулярности исходной системы:

\begin{consequence}
	Если система билинейных уравнений не регулярна, то определитель матрицы $M$ равен нулю:
	$$
	\text{исходная система не регулярна} \implies \det M = 0 
	$$
\end{consequence}

\noindentПриведем пример:
\begin{equation*}
	\begin{cases}
		r_1 s_1 = 0 \\
		r_2 s_2 = 0 \\
		6 r_1 s_2 + 8 r_2 s_1 = 0 \\
	\end{cases}
\end{equation*}
Для этой матрицы построим мономно-минорную матрицу $M$ и найдем её определитель:
$$
	M =
	\begin{pmatrix}
	0 & -1 & \;\;\,0 \\
	0 & \;\;\,0 & -6 \\
	8 & \;\;\,0 & \;\;\,0 
	\end{pmatrix}
	\implies 
	\det M = 48 \ne 0
$$
Таким образом, эта система является регулярной. Действительно, эта система имеет только тривиальные решения (это можно легко проверить с помощью простого перебора вариантов).

\subsection{Приведение к каноническому виду}
Перепишем систему уравнений (\ref{eq:trivial}) в виде (\ref{eq:forms}):
\begin{equation}
	\begin{cases}
		\mathbf{r}^T A_1 \mathbf{s} = 0 \\
		\mathbf{r}^T A_2 \mathbf{s} = 0 \\
		\mathbf{r}^T A_3 \mathbf{s} = 0 \\						
	\end{cases}
\end{equation}
Пусть, не теряя общности, матрица билинейной формы $A_1$ является невырожденной. В таком случае, 
мы можем ввести замену переменных, чтобы улучшить вид исходной системы:
$$
	\mathbf{r} = C \mathbf{r}^*, \mathbf{s} = D \mathbf{s}^*,~\textrm{где}~C, R \in \mathbb{R}^{2 
	\times 2}  
$$
Положим $C = A_1^{-T}$, $D = E$. Тогда $A_1 = {(A_1^{-T})}^T \cdot A_1 \cdot E = E$. 
Далее, предлагается ввести еще одну замену, которая сохранит тривиальный вид формы $A_1$ и 
улучшит матрицу билинейной формы $A_2$. 


\end{document}