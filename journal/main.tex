% XeLaTeX can use any Mac OS X font. See the setromanfont command below.
% Input to XeLaTeX is full Unicode, so Unicode characters can be typed directly into the source.

% The next lines tell TeXShop to typeset with xelatex, and to open and save the source with Unicode encoding.

%!TEX TS-program = xelatex
%!TEX encoding = UTF-8 Unicode
%!BIB program = biber

\documentclass[10pt]{article}
\usepackage{geometry}
\geometry{a5paper, left=22.2mm, top=20.0mm, bottom=20.0mm, right=11.5mm}

\usepackage{graphicx}
\usepackage{amssymb, amsmath, amsthm} 
\usepackage[english, russian]{babel}

%\numberwithin{equation}{section}

\usepackage{setspace}
\onehalfspacing

\usepackage[style=gost-numeric,
	language=auto,
	autolang=other,
	sorting=none,
	backend=biber
]{biblatex}
\addbibresource{../bib_files/bibliography.bib}

\usepackage{fontspec,xltxtra,xunicode}
\defaultfontfeatures{Mapping=tex-text}
\usepackage{mathptmx}
\usepackage{fontspec}
\setmainfont{Times New Roman}
\setlength\parindent{0.75cm}

\usepackage{lipsum}
\usepackage{indentfirst}
\usepackage{titlesec}

\newtheorem{theorem}{Теорема}
\newtheorem{lemma}{Лемма}
\newtheorem{consequence}{Следствие}

\titleformat{\section}[block]
{\normalfont\bfseries\filcenter}
{\thesection.}{1em}{}

%\title{Системы Билинейных Уравнений}
%\author
%{
%	Выполнил: студент 3 курса Морозов Евгений \\
%	Руководитель: А.В. Лобода
%}
%\date{}

\begin{document}

%\maketitle

\begin{center}
\fontsize{11pt}{16pt}\selectfont\textbf{СИСТЕМЫ БИЛИНЕЙНЫХ УРАВНЕНИЙ}
\end{center}

\begin{flushright}
Е.Ю. Морозов \\ 
Студент \\
А.В. Лобода \\ 
Профессор
\end{flushright}

\section*{Введение}

Для билинейных систем не существует теории, которая бы позволила решать их в 
общем виде. Тем не менее, в некоторых математических задачах, к примеру, в задаче описания 
аффинно-однородных поверхностей в пространстве $ \mathbb{C}^3 $, возникает такая 
потребность. В однородных билинейных системах, которые появляются в указанной задаче, 
особый интерес представляет наличие ненулевых решений. В данной работе рассматриваются частные случаи билинейных систем и выводится критерий, позволяющий определить, имеет ли система ненулевые решения.

\section{Постановка задачи}
Система билинейных уравнений (или билинейная система) "--- это система следующего вида:
\begin{equation}\label{eq:bilinear_system}
	\sum_{j=1}^{m} \sum_{k=1}^{n} a^{i}_{jk} r_j s_k = d_i,~i=\overline{1,l},
\end{equation}
где $m,n$ "--- число неизвестных в векторах $\mathbf{r}$ и $\mathbf{s}$, $l$ "--- число уравнений и $a^i_{jk} \in \mathbb{R}$ "--- действительные числовые коэффициенты~\cite{stanford}. 
Для такой системы введём компактную запись её параметров $(l; m, n)$. 

Билинейную систему можно записать в альтернативном виде: 
\begin{equation}\label{eq:alternative_system}
	\mathbf{r}^T \cdot A_i \cdot \mathbf{s} = d_i,
\end{equation} 
где $\mathbf{r} \in \mathbb{R}^{m}$, $\mathbf{s} \in \mathbb{R}^{n}$ "--- наборы 
неизвестных, $A_i \in \mathbb{R}^{m \times n}$ "--- матрица билинейной формы $i$-го 
уравнения. 

Для каждой билинейной системы можно составить \textit{обощенную матрицу системы}, составленную из развернутых в строки матриц билинейных форм каждого уравнения:
$$
	A = 
	\begin{pmatrix}
	a^{1}_{11} & \hdots  & a^{1}_{1n}  & \hdots &  a^{1}_{mn} \\ 
	\vdots & \ddots  & \ddots & \ddots  & \vdots \\
	a^{l}_{11} & \hdots  & a^{l}_{1n}  & \hdots  & a^{l}_{mn} \\ 
	\end{pmatrix} \in \mathbb{R}^{l \times m \cdot n}
$$ 
Ранг системы билинейных форм (\ref{eq:bilinear_system}) совпадает с рангом этой матрицы. В этой работе будут рассматриваться только системы полного ранга.

Билинейная система называется \textit{однородной}, если правая часть равна нулю, т.е. $d_i 
= 0$. У таких систем всегда есть тривиальные решения "--- решения, при которых один из 
наборов в переменных является нулевым, т.е. $\mathbf{s} = \mathbf{0}$ или $\mathbf{r} = 
\mathbf{0}$.

В данной работе будет рассматриваться вопрос наличия именно нетривиальных решений в 
однородных системах билинейных уравнений. Как правило, системы с большим количеством уравнений имеют только тривиальные решения, поэтому системы, имеющие нетривиальные решения будем называть \textit{нерегулярными}.

В уже упомянутой задаче описания аффинно-однородных поверхностей в пространстве $ \mathbb{C}^3 $, возникает потребность в установлении факта регулярности системы. Один из возникающих 
сложных случаев связан с 
$
	m = n,~l = 2m - 1.
$
Особый интерес в уже упомянутой задаче представляют именно нерегулярные системы и в основном случае $(15; 8, 8)$ нетривиальные решения пока не описаны. В связи с этим, в данной работе изучаются более простые случаи. Самый простой из них "--- $(1; 1, 1)$, являющийся тривиальным, и рассматривать его отдельно не имеет смысла.

\section{Случай (3;2,2)-систем}
В следующем случае $(3; 2, 2)$ возникает система вида:
\begin{equation}\label{eq:simple_bilinear}
	\begin{cases}
		a_{1,1} r_1 s_1 + a_{1,2} r_1 s_2 + a_{1,3} r_2 s_1 + a_{1,4} r_2 s_2 = 0 \\
		a_{2,1} r_1 s_1 + a_{2,2} r_1 s_2 + a_{2,3} r_2 s_1 + a_{2,4} r_2 s_2 = 0 \\
		a_{3,1} r_1 s_1 + a_{3,2} r_1 s_2 + a_{3,3} r_2 s_1 + a_{3,4} r_2 s_2 = 0 \\
	\end{cases}
\end{equation}
Рассмотрим набор переменных $\mathbf{r}$. Вынося $r_1$ и $r_2$ как общие множители и переписывая систему (\ref{eq:simple_bilinear}) в матричном виде, получаем: 
\begin{equation}\label{eq:r_bilinear}
        \underbrace{
        \begin{pmatrix}
                a_{11} s_1 + a_{12} s_2 & a_{13} s_1 + a_{14} s_2 \\
                a_{21} s_1 + a_{22} s_2 & a_{23} s_1 + a_{24} s_2 \\
                a_{31} s_1 + a_{32} s_2 & a_{33} s_1 + a_{34} s_2 \\
        \end{pmatrix}}_{B}
        \times
        \begin{pmatrix}
                r_1 \\
                r_2
        \end{pmatrix}=
        \begin{pmatrix}
                0 \\
                0 \\
                0
        \end{pmatrix}
\end{equation}
Ясно, что при фиксированных значениях $\mathbf{s}$ эта система будет иметь 
нетривиальные по $\mathbf{r}$ решения, если ранг матрицы $B$ не полон (в противном случае 
система имеет единственное решение "--- $\mathbf{r} = \mathbf{0}$)~\cite{costrikin_va1}. 
Составим из трёх миноров второго порядка матрицы $B$ новую систему уравнений:
\begin{equation}
	\begin{cases} 
		M_1 = Q(\mathbf{s}) = 0 \\
		M_2 = Q(\mathbf{s}) = 0 \\
		M_3 = Q(\mathbf{s}) = 0 \\
	\end{cases},
\end{equation}
где $Q_1(\mathbf{s})$, $Q_2(\mathbf{s})$ и $Q_3(\mathbf{s})$ "--- квадратичные формы от $
\mathbf{s}$. Эту систему можно переписать в матричном виде, выделяя в качестве вектора 
неизвестных мономы второго порядка относительно $s$: 
\begin{equation}\label{eq:monoms_system}
	M \times  
	\begin{pmatrix}
		s_1s_2 \\ 
		s_1^2 \\ 
		s_2^2
	\end{pmatrix} = 
	\begin{pmatrix}
		0 \\
		0 \\
		0
	\end{pmatrix}
\end{equation}
Рассмотрим случай, когда ранг матрицы $M$ полон, т.е. матрица $M$ невырожденна. Тогда система (\ref{eq:monoms_system}) имеет единственное решение. Как следствие, $\mathbf{s} = \mathbf{0}$. Это означает, что система (\ref{eq:simple_bilinear}) имеет только тривиальные решения. Отсюда вытекает \textit{достаточное условие регулярности}: 
\begin{theorem}(достаточное условие регулярности):
	если определитель матрицы $M$ не равен нулю, то билинейная $(3;2,2)$-система является
	регулярной. 
\end{theorem}
Примечательно, что аналогичные рассуждения относительно набора переменных $\mathbf{s}$ 
приводят к точно такому же определителю. Следствием из теоремы 1 является \textit{необходимое условие нерегулярности}:
\begin{consequence}(необходимое условие нерегулярности):
если билинейная $(3;2,2)$-система является нерегулярной, то определитель матрицы $M$ равен нулю.
\end{consequence}
Определитель матрицы $M$ имеет шестой порядок, зависит от двенадцати коэффициентов и выглядит довольно громоздко. Пусть $d_{i,j,k}$ "--- минор второго порядка в матрице системы (\ref{eq:simple_bilinear}), полученный вычеркиванием $i$-й строки и столбцов $j$ и $k$. Тогда определитель матрицы $M$ будет иметь вид:
\begin{align*}
 	\det M & = (d_{2,1,4} d_{3,2,4} + d_{2,2,3} d_{3,2,4} - d_{2,2,4} d_{3,1,4} - 
 	d_{2,2,4} d_{3,2,3}) d_{1,1,3} + \\
 	& + (d_{1,2,4} d_{3,1,4} + d_{1,2,4} d_{3,2,3} - d_{1,1,4} d_{3,2,4} -	
	d_{1,2,3} d_{3,2,4}) d_{2,1,3} + \\		 
 	& + (d_{1,1,4} d_{2,2,4} + d_{1,2,3}d_{2,2,4} - d_{1,2,4}d_{2,1,4} - 
	d_{1,2,4} d_{2,2,3}) d_{3,1,3}
\end{align*} 

\section{Критерий регулярности для (3;2,2)-систем}
Перепишем систему (\ref{eq:simple_bilinear}) в виде (\ref{eq:alternative_system}): 
\begin{equation}\label{eq:simple_forms}
	\begin{cases}
		\mathbf{r}^T \cdot A_1 \cdot \mathbf{s} = 0 \\ 
		\mathbf{r}^T \cdot A_2 \cdot \mathbf{s} = 0 \\ 
		\mathbf{r}^T \cdot A_3 \cdot \mathbf{s} = 0 \\ 
	\end{cases}
\end{equation}
Обозначим \textit{допустимые преобразования}, которые не изменяют факта регулярности
произвольной $(l; m, n)$ -системы: 
\begin{itemize}
 	\item гауссовы преобразования уравнений в системе, 
	
	\item замены переменных $\mathbf{r} = C \cdot \mathbf{r}^*$, $\mathbf{s} = D \cdot 
	\mathbf{s}^*$ (матрицы $C$ и $D$ невырожденны), которые изменяют матрицу билинейной 
	формы $A_i$ по закону:
	$$
		A^{*}_{i} = C^T \cdot A_i \cdot D
	$$
\end{itemize}
Для дальнейших преобразований потребуется хотя бы одно уравнение с невырожденной матрицей 
билинейной формы. 
\begin{lemma}
	Если $(3;2,2)$-система является системой полного ранга, то существует гауссово 
	преобразование билинейных форм из этой системы, приводящее к системе с невырожденной 
	матрицей билинейной формы.
\end{lemma}

Используя эту лемму, можно получить систему с одной невырожденной матрицей 
билинейной формы. Пусть этой матрицей будет матрица $A_1$. Тогда <<улучшим>> вид этой матрицы, вводя замену
$
	\mathbf{r} =  A_1^T \mathbf{r}^*, \mathbf{s} = \mathbf{s}^*
$.
В результате получаем
$$ 
	A_1^* = (A_1^{-T})^T \cdot A_1 \cdot E = E
$$
Произведём еще одну замену так, чтобы <<улучшить>> вид формы $A_2^*$, полученной после 
первой замены, сохраняя тривиальный (единичный) вид формы $A_1$. Вид матрицы $A_2^{'}$ 
будет зависеть от собственных значений и собственных векторов матрицы $A_2^*$. Для матрицы 
$2 \times 2$ возможны следующие три случая~\cite{costrikin_va2}: 
\begin{itemize}
	\item Различные собственные значения (простой спектр): $\lambda_1 \ne \lambda_2$;
	\item Кратные собственные значения: $\lambda_1 = \lambda_2$
	\item Комплексные собственные значения: $\lambda_{1,2} = \alpha \pm \imath \beta,~\beta 
	\ne 0$.
\end{itemize}
Рассматривая все три случая, можно  сформулировать промежуточную теорему об упрощенном виде системы (\ref{eq:simple_forms}):
\begin{theorem}
Любая билинейная $(3;2,2)$-система полного ранга приводима допустимыми преобразованиями к 
одному из трёх упрощенных видов: 
	\begin{equation*}\
		1)\begin{cases}
			r_1 s_1 + r_2 s_2 = 0 \\ 
			r_1 s_2 - r_2 s_1 = 0 \\ 
			s_1 (a'_{1} r_1 + a'_{2} r_2) = 0
		\end{cases},
		~
		2)\begin{cases}
			r_1 s_2 = 0 \\
			r_1 s_1 + r_2 s_2 = 0 \\
			s_1 (a'_{3} r_1 + a'_{4} r_2) = 0
		\end{cases},
	\end{equation*}
	\begin{equation}\label{eq:simplified_equations}
		3)\begin{cases}
		 	r_1 s_1 = 0 \\
		 	r_2 s_2 = 0 \\
		 	r_1 s_2 \cos \varphi + r_2 s_1 \sin \varphi = 0 &
		\end{cases},
	\end{equation}
	где $a'_1, a'_2$ и $a'_3, a'_4$ одновременно не равны нулю, $\varphi$ "--- некоторые коэффициенты, 	полученные после произведённых преобразований.
\end{theorem}
Рассмотрим условия, при которых упрощенные виды (\ref{eq:simplified_equations}) могут иметь 
нетривиальные решения. Воспользуемся необходимым условием нерегулярности, сформулированным 
в следствии 1 и вычислим определители соответствующих матриц для каждого вида, приравнивая 
их к нулю: 
\begin{equation}\label{eq:constraints}
	1)~\det M = -({a'_1}^2 + {a'_2}^2)=0 \quad  2)~\det M = a'_{4}=0 \quad 3)\det M = \cos \varphi \cdot \sin \varphi =0
\end{equation}
Накладывая ограничения (\ref{eq:constraints}) на коэффициенты в упрощенных формах 
(\ref{eq:simplified_equations}), мы получим \textit{канонические} виды, которые имеют 
нетривиальные решения. Отсюда формулируется необходимое и достаточное условие 
нерегулярности для билинейной системы полного ранга (критерий нерегулярности): 
\begin{theorem}(критерий нерегулярности)
	Билинейная $(3;2,2)$-система полного ранга нерегулярна тогда и только тогда, когда она 
	приводима к одному из двух канонических видов: 
	\begin{align*}
		&
		1)\begin{cases}
		 	r_1 s_1 = 0 \\
		 	r_2 s_2 = 0 \\
		 	r_1 s_2 = 0 &
		\end{cases}
		&
		2)\begin{cases}
			r_1 s_1 = 0 \\
		 	r_2 s_2 = 0 \\
		 	r_2 s_1 = 0 &
		\end{cases}
	\end{align*}
\end{theorem} 

\section{Заключение}
В этой работе была рассмотрена проблема определения нерегулярности однородной системы 
билинейных уравнений. В общем случае эта задача слишком сложна для рассмотрения, поэтому 
были исследованы более простые случаи. Досконально был проанализирован случай $(3;2,2)$-системы и был выведен критерий нерегулярности для случая системы полного ранга, который позволяет, не решая систему, определить факт её нерегулярности. 

\printbibliography

\end{document}  