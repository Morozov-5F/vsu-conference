\documentclass[russian,hyperref={unicode}]{beamer}
% Encoding and locale packages
\usepackage[T2A]{fontenc}
\usepackage[utf8]{inputenc}
\usepackage[russian]{babel}
% Other packages
\usepackage{amsmath, amssymb, amsthm}
% \usepackage{beamerthemesplit} // Activate for custom appearance


% Theme selection 
\usetheme[sectionpage=none, numbering=fraction]{metropolis}
%\usecolortheme{beaver}

% \setbeamertemplate{navigation symbols}
% {%
    % \usebeamerfont{footline}%
    % \usebeamercolor[fg]{footline}%
%    \hspace{1em}%
    % \insertframenumber/\inserttotalframenumber
% }	

\title{Системы билинейных уравнений}
\institute
{
	Воронежский Государственный Университет \\
	Факультет Компьютерных Наук \\
	Кафедра Цифровых Технологий
}
\author
{
	Выполнил: студент 3 курса Морозов Евгений \\
	Руководитель: А.В. Лобода
}
\date{19 мая 2016 г.}

\begin{document}

\frame{\titlepage}

%\section[Содержание]{}
%\frame
%{
%	\frametitle{Содержание}
%	\tableofcontents
%}

\section{Общий случай}
\subsection{Понятие системы билинейных уравнений}
\frame
{
  \frametitle{Основные понятия}
  
	\begin{itemize}
		\item \textbf{Система билинейных уравнений} "--- это система уравнений следующего 
			вида:
			\begin{align*}
				&\sum_{k,\,j}a^{i}_{jk} r_j s_k = d_i,~j=1 \dots m,~k=1 \dots n, \\
				& i = 1 \dots l \text{ "--- номер уравнения в системе.}
			\end{align*}
		\item Компактная запись параметров системы "--- $(l; m, n)$
		\item Система называется \textbf{однородной}, если $\textbf{d} = \textbf{0}$. 
	 	\item В однородных 	системах всегда имеются \textbf{тривиальные} решения: 
			$ \textbf{s} = \textbf{0}$ или $\textbf{r} = \textbf{0} $.
		\item В данной работе изучается вопрос о наличии нетривиальных решений у 
		однородных систем.
		\item Систему, имеющую нетривиальные решения будем называть 
		\textbf{нерегулярной}.
	\end{itemize}
}
\subsection{Связь с задачей об однородности}
\frame
{
	\frametitle{Задача об однородности}
	
	Системы однородных билинейных уравнений  возникают, например,
	в задаче об аффинной однородности поверхностей. Один из возникающих сложных случаев 
	связан с 
	$$m = n,\,l = 2m - 1,~\text{в частности, } (15; 8, 8).$$ 
	Особый интерес в этой задаче представляют именно нерегулярные системы.
	\begin{itemize}
		\item В основном случае $(15; 8, 8)$ "--- полная картина пока не получена.
		\item В связи с этим в работе исследуется более простые случаи.
	\end{itemize}
}
\section{Частный случай}
\subsection{Основные понятия}
\frame
{
	\frametitle{Случай $(3; 2, 2)$: основные понятия}
	Любую билинейную систему можно записать в ином виде: 
	\begin{align*}
	    & \mathbf{r}^T \cdot A_i \cdot \mathbf{s} = 0,~i = 1 \dots l \\
	    & A_i = \left[ a^i_{jk} \right]  \in \mathbb{R}^{m \times n}
	\end{align*}
	Тогда составим матрицу $A \in \mathbb{R}^{l \times m \cdot n}$ из развернутых в 
	строки матриц билинейных форм $A_i$ и назовем \textbf{матрицей системы}. Для случая 
	$(3;2,2)$:
	$$
	A = 
	\begin{pmatrix}
		a^1_{11} & a^1_{12} & a^1_{21} & a^1_{22} \\
		a^2_{11} & a^2_{12} & a^2_{21} & a^2_{22} \\
		a^3_{11} & a^3_{12} & a^3_{21} & a^3_{22} \\
	\end{pmatrix} = 
	\begin{pmatrix}
		a_{11} & a_{12} & a_{13} & a_{34} \\
		a_{21} & a_{22} & a_{23} & a_{34} \\
		a_{31} & a_{32} & a_{33} & a_{34} \\
	\end{pmatrix}
	$$
	Далее будем рассматривать только системы, имеющие полный ранг матрицы $A$, т.е. $
	\text{rank}\,A = 3$. 
}
\subsection{Постановка задачи}
\frame
{
	\frametitle{Случай $(3; 2, 2)$: исследование решений}
	Любая билинейная система является линейной по одному из наборов переменных. Для 
	случая $(3;2, 2)$:
	$$
		\underbrace{
		\begin{pmatrix}
			a_{1 1} s_1 + a_{1 2} s_2 & a_{1 3} s_1 + a_{1 4} s_2 \\
			a_{2 1} s_1 + a_{2 2} s_2 & a_{2 3} s_1 + a_{2 4} s_2 \\
			a_{3 1} s_1 + a_{3 2} s_2 & a_{3 3} s_1 + a_{3 4} s_2 \\
		\end{pmatrix}}_{B}
		\times
		\begin{pmatrix}
			r_1 \\
			r_2
		\end{pmatrix}=
		\begin{pmatrix}
			0 \\
			0 \\
			0
		\end{pmatrix}
	$$
	При фиксированных $\mathbf{s}$ эта система имеет нетривиальные по $\mathbf{r}$ 
	решения, если ранг матрицы $B$ не полон.
	В тако случае составим новую систему уравнений из трех миноров второго порядка 
	матрицы $B$, приравняв их к нулю:
	$$
	\begin{cases} 
		M_1 = Q_1(\mathbf{s}) = 0 \\
		M_2 = Q_2(\mathbf{s}) = 0 \\
		M_3 = Q_3(\mathbf{s}) = 0
	\end{cases}
	~\backsim~	
	M \times 
	\begin{pmatrix}
	s_1 s_2 \\
	s_1 ^ 2 \\
	s_2 ^ 2
	\end{pmatrix}
	= 
	\begin{pmatrix}
	0 \\
	0 \\
	0
	\end{pmatrix}
	$$
}
\subsection{Достаточное условие регулярности системы}
\frame
{
	\frametitle{Достаточное условие регулярности системы}
	Для нахождения нетривиальных по $\mathbf{r}$ решений, нужно найти решения предыдущей 
	системы уравнений относительно $\mathbf{s}$. Если $\det M \ne 0$, то решения только 
	тривиальные. При аналогичных рассуждениях относительно $\mathbf{s}$, явный вид 
	определителя совпадает, что дает \textbf{достаточное условие регулярности}.  
		
	Пусть $d_{k, m, n}$ "--- минор второго порядка в матрице $A$, получаемый 
	вычеркиванием строки $k$ и столбцов $m$ и $n$.
	
	\textbf{Теорема 1:}
	Если определитель 
	\begin{align*}
	 	\det M & = (d_{2,1,4} d_{3,2,4} + d_{2,2,3} d_{3,2,4} - d_{2,2,4} d_{3,1,4} - 
	 	d_{2,2,4} d_{3,2,3}) d_{1,1,3} + \\
	 	& + (d_{1,2,4} d_{3,1,4} + d_{1,2,4} d_{3,2,3} - d_{1,1,4} d_{3,2,4} -	
		d_{1,2,3} d_{3,2,4}) d_{2,1,3} + \\		 
	 	& + (d_{1,1,4} d_{2,2,4} + d_{1,2,3}d_{2,2,4} - d_{1,2,4}d_{2,1,4} - 
		d_{1,2,4} d_{2,2,3}) d_{3,1,3}
	\end{align*} 
	не равен нулю, то $(3; 2, 2)$-система является \textit{регулярной}.
}
\subsection{Другие подходы}
\frame
{
	\frametitle{Преобразования билинейной системы}
	Обозначим \textbf{допустимые} преобразования, которые не изменяют факт регулярности 
	произвольной $(l; m, n)$-системы: 
	\begin{itemize}
		\item линейные комбинации уравнений;
		\item замены переменных $
			\mathbf{r} = C \mathbf{r}^*, \mathbf{s} = D \mathbf{s}^* 
		$, изменяющие матрицу по следующему закону: 
		$$
			A^* = C^T \cdot A \cdot D,~A \in \mathbb{R}^{m \times n},
		$$
		$C$, $D$ "--- невырожденные матрицы соответствующих размерностей. 
	\end{itemize}
	\begin{block}{Лемма 1 (о невырожденной форме):}
		Если ранг матрицы $(3; 2,2)$-системы полон, то существует линейная комбинация 
		матриц билинейных форм системы, имеющая ненулевой определитель.
	\end{block}
}
\subsubsection{Упрощенный вид системы}
\frame
{
	\frametitle{Упрощенный вид системы}
	\begin{block}{Теорема 2 (промежуточная):}
	Билинейная $(3; 2, 2)$-система полного ранга приводима допустимыми 
	преобразованиями к одному из трёх упрощенных видов: 
	\begin{align*}
		&
		1)\begin{cases}
			r_1 s_1 + r_2 s_2 = 0 \\ 
			r_1 s_2 - r_2 s_1 = 0 \\ 
			s_1 (a'_{1} r_1 + a'_{2} r_2) = 0
		\end{cases}
		&
		2)\begin{cases}
			r_1 s_2 = 0 \\
			r_1 s_1 + r_2 s_2 = 0 \\
			s_1 (a'_{1} r_1 + a'_{2} r_2) = 0
		\end{cases} 
	\end{align*}
	$$
		3)\begin{cases}
		 	r_1 s_1 = 0 \\
		 	r_2 s_2 = 0 \\
		 	r_1 s_2 \cos \varphi + r_2 s_1 \sin \varphi = 0 &
		\end{cases}
	$$
	\end{block}
}
\subsubsection{Критерий нерегулярности}
\frame
{
	\frametitle{Критерий нерегулярности}
	\begin{block}{Теорема 3 (критерий нерегулярности):}
	Билинейная $(3;2,2)$-система полного ранга нерегулярна тогда и только тогда, когда 
	она приводима к одному из двух канонических видов:
	
	\begin{align*}
		&
		1)\begin{cases}
		 	r_1 s_1 = 0 \\
		 	r_2 s_2 = 0 \\
		 	r_1 s_2 = 0 &
		\end{cases}
		&
		2)\begin{cases}
			r_1 s_1 = 0 \\
		 	r_2 s_2 = 0 \\
		 	r_2 s_1 = 0 &
		\end{cases}
	\end{align*}
	\end{block}
}
\section{Метод мономно-минорной матрицы}
\subsection{Описание метода}
\frame
{
	\frametitle{Метод мономов и миноров (МММ)}
	Этот метод предлагается для исследования 
	нетривиальных решений в однородных системах билинейных уравнений.

	Метод заключается в построении специальной системы \textit{линейных} 
	уравнений на основе исходной. Решая новую систему, можно получить нетривиальные 
	решения соответствующей билинейной системы или доказать её регулярность.
	
	Этот метод позволил полностью описать нетривиальные решения $(7; 4, 4)$- и $(9; 
	5,5)$-систем.

	\begin{alertblock}{Замечание:}
		К сожалению, этот метод не дает должных результатов на больших системах. В случае 
		$(15; 8, 8)$ возникает СЛАУ $6435 \times 6435$, которую необходимо решить 
		аналитически. С этой задачей пока что не справился ни один математический 
		пакет.
	\end{alertblock}
}
\section{Литература}
\frame
{
	\frametitle{Литература}
	\begin{thebibliography}{10}
	\beamertemplatebookbibitems
  		\bibitem{kostrikin_va1}
    	Кострикин А.И. 
    	\newblock {\em Введение в алгебру. Часть I. Основы алгебры}.
    	\newblock М.: Физико-математическая литература, 2000.
	\beamertemplatearticlebibitems
		\bibitem{stanford}
    	S. Cohen, C. Tomasi
    	\newblock {\em Systems of Bilinear Equations}.
    	\newblock Computer Science Department, Stanford University, 1998
    \end{thebibliography}
}
\frame
{
	\begin{center}
		\huge Спасибо за внимание
	\end{center}
}

\end{document}