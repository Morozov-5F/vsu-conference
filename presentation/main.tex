\documentclass[russian,hyperref={unicode}]{beamer}
% Encoding and locale packages
\usepackage[T2A]{fontenc}
\usepackage[utf8]{inputenc}
\usepackage[russian]{babel}
% Other packages
\usepackage{amsmath, amssymb, amsthm}
% \usepackage{beamerthemesplit} // Activate for custom appearance


% Theme selection 
\usetheme[sectionpage=none, numbering=fraction]{metropolis}
%\usecolortheme{beaver}

% \setbeamertemplate{navigation symbols}
% {%
    % \usebeamerfont{footline}%
    % \usebeamercolor[fg]{footline}%
%    \hspace{1em}%
    % \insertframenumber/\inserttotalframenumber
% }	

\title{Системы билинейных уравнений}
\institute
{
	Воронежский Государственный Университет \\
	Факультет Компьютерных Наук
}
\author{Морозов Евгений}
\date{19 мая 2016 г.}

\begin{document}

\frame{\titlepage}

\section[Содержание]{}
\frame
{
	\frametitle{Содержание}
	\tableofcontents
}

\section{Общий случай}
\subsection{Понятие системы билинейных уравнений}
\frame
{
  \frametitle{Основные понятия}
  
	\begin{itemize}
		\item \textbf{Система билинейных уравнений} "--- это система уравнений следующего вида:
		$$
			\sum_{j = 1}^{m} \sum_{k = 1}^{n} a_{i j k} r_j s_k = d_i,~\text{для } i = 
			1,\dots,l
		$$
		\item Компактная запись параметров системы "--- $(l; m, n)$
		\item Система называется \textbf{однородной}, если $\textbf{d} = \textbf{0}$. 
	 	\item В однородных 	системах всегда имеют место \textbf{тривиальные} решения: $\textbf{s} =  
			  \textbf{0}$ или $\textbf{r} = \textbf{0}$.
		\item Система, имеющие только тривиальные решения называется \textbf{регулярной}.
		\item В данной работе рассматриваются только однородные системы.
	\end{itemize}
}
\subsection{Связь с задачей об однородности}
\frame
{
	\frametitle{Задача об однородности}
	
	В задаче об однородности аффинных поверхностей возникают 
	системы однородных билинейных уравнений особого вида : 
	$$l = m + n - 1,\, m = n.$$ 
	Особый интерес в этой задаче представляют нерегулярные системы.
	Примеры возникающих систем:
	\begin{itemize}
		\item Случай $(15; 8, 8)$ "--- все стандартные методы решения нелинейных систем 
		(например, базис Грёбнера) не дают результатов. 
		\item В связи с этим имеет смысл исследовать более простой, <<модельный>> случай 
		$(3;2,2)$, чтобы понять природу нерегулярных систем.
	\end{itemize}
}
\section{Частный случай}
\subsection{Постановка задачи}
\frame
{
	\frametitle{Случай $(3; 2, 2)$}
	Любая билинейная система является линейной по одному из наборов переменных. Для случая 	
	$(3;2, 2)$:
	$$
		\underbrace{
		\begin{pmatrix}
			a_{1 1} s_1 + a_{1 2} s_2 & a_{1 3} s_1 + a_{1 4} s_2 \\
			a_{2 1} s_1 + a_{2 2} s_2 & a_{2 3} s_1 + a_{2 4} s_2 \\
			a_{3 1} s_1 + a_{3 2} s_2 & a_{3 3} s_1 + a_{3 4} s_2 \\
		\end{pmatrix}}_{A}
		\times
		\begin{pmatrix}
			r_1 \\
			r_2
		\end{pmatrix}=
		\begin{pmatrix}
			0 \\
			0 \\
			0
		\end{pmatrix}
	$$
	Эта система имеет нетривиальные решения, если $\text{rank}\,A 
	\ne 3$, что означает, что все миноры третьего порядка в матрице $A$ равны нулю.
	\begin{alertblock}{Замечание:}
		Понятие ранга справедливо при фиксированных $\mathbf{s}$. Для их нахождения составим 
		систему уравнений из миноров третьего порядка матрицы $A$.
	\end{alertblock}
}
\subsection{Достаточное условие регулярности системы}
\frame
{
	\frametitle{Достаточное условие регулярности системы}
	Составим матрицу $A \in \mathbb{R}^{3 \times 4}$ из коэффициентов системы так, чтобы 
	$A_{i,j} = a_{i,j}$. 
	
	Пусть $d_{k, m, n}$ "--- минор второго порядка в матрице $A$, получаемый вычеркиванием строки $k$ 
	и столбцов $m$ и $n$.
	
	Тогда достаточное условие регулярности системы формулируется так: 
	\begin{block}{Теорема:}
		Если определитель следующего вида не равен нулю, то система является 
		\textit{регулярной}:
		\begin{align*}
		 	\det M & = (d_{2,1,4} d_{3,2,4} + d_{2,2,3} d_{3,2,4} - d_{2,2,4} d_{3,1,4} - 
		 	d_{2,2,4} d_{3,2,3}) d_{1,1,3} + \\
		 	& + (d_{1,2,4} d_{3,1,4} + d_{1,2,4} d_{3,2,3} - d_{1,1,4} d_{3,2,4} - d_{1,2,3}
		 	d_{3,2,4}) d_{2,1,3} + \\		 
		 	& + (d_{1,1,4} d_{2,2,4} + d_{1,2,3}d_{2,2,4} - d_{1,2,4}d_{2,1,4} - d_{1,2,4}
		 	d_{2,2,3}) d_{3,1,3}
		\end{align*} 
	\end{block}
}
\subsection{Другие подходы}
\frame
{
	\frametitle{Канонический вид системы}
	\begin{block}{Теорема:}
	Пусть дана система билинейных уравнений с параметрами $(3, 2, 2)$ и матрица одной из билинейных форм невырождена. Тогда эта система приводима к одному из четырех канонических видов: 
	\begin{align*}
		&
		\begin{cases}
		 	r_1 s_1 = 0 \\
		 	r_2 s_2 = 0 \\
		 	r_1 s_2 \cos \varphi + r_2 s_1 \sin \varphi = 0 &
		\end{cases} &, &
		\begin{cases}
			r_1 s_1 + r_2 s_2 = 0 \\
			a'_{1}r_1 s_1 + a'_2 r_1 s_2 + a'_3 r_2 s_1 = 0
		\end{cases},
		\\
		&
		\begin{cases}
			r_1 s_1 + r_2 s_2 = 0 \\
			r_1 s_1 = 0 \\
			s_1 (a'_{1} r_1 + a'_{2} r_2) = 0
		\end{cases} &, &
		\begin{cases}
			r_1 s_1 + r_2 s_2 = 0 \\ 
			r_1 s_2 - r_2 s_1 = 0 \\ 
			s_1 (a'_{1} r_1 + a'_{2} r_2) = 0
		\end{cases}
	\end{align*}
	\end{block}
}
\section{Метод мономно-минорной матрицы}
\subsection{Описание метода}
\frame
{
	\frametitle{Метод мономно-минорной матрицы (МММ)}
	Этот метод был введен проф. А.В. Лободой для исследования 
	нетривиальных решений в однородных системах билинейных уравнений.

	Метод заключается в построении особой системы \textit{линейных} 
	уравнений на основе исходной. Решая новую систему, можно получить нетривиальные решения  соответствующей билинейной системы или доказать её регулярность.

	\begin{alertblock}{Замечание:}
		К сожалению, этот метод не дает должных результатов на больших системах. В случае $(15; 8, 8)$ возникакет матрица $6435 \times 6435$, которую необходимо решить аналитически. С этой задачей не справился ни один математический пакет.
	\end{alertblock}
}
\section{Литература}
\frame
{
	\frametitle{Литература}
	\begin{thebibliography}{10}
	\beamertemplatebookbibitems
  		\bibitem{kostrikin_va1}
    	Кострикин А.И. 
    	\newblock {\em Введение в алгебру. Часть I. Основы алгебры}.
    	\newblock М.: Физико-математическая литература, 2000.
	\beamertemplatearticlebibitems
		\bibitem{stanford}
    	S. Cohen, C. Tomasi
    	\newblock {\em Systems of Bilinear Equations}.
    	\newblock Computer Science Department, Stanford University
    \end{thebibliography}
}
\frame
{
	\begin{center}
		\huge Спасибо за внимание!
	\end{center}
}

\end{document}