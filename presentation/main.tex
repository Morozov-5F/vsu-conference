\documentclass{beamer}
% Encoding and locale packages
\usepackage[T2A]{fontenc}
\usepackage[utf8]{inputenc}
\usepackage[english, russian]{babel}
% Other packages
\usepackage{amsmath, amssymb, amsthm}
% \usepackage{beamerthemesplit} // Activate for custom appearance

% Theme selection 
\usetheme{default}
\usecolortheme{beaver}

\title{Системы билинейных уравнений}
\institute
{
	Воронежский Государственный Университет \\
	Факультет Компьютерных Наук
}
\author{Морозов Евгений}
\date{19 мая 2016 г.}

\begin{document}

\frame{\titlepage}

\section[Outline]{}
\frame{\tableofcontents}

\section{Общий случай}
\subsection{Понятие системы билинейных уравнений}
\frame
{
  \frametitle{Основные понятия}
  
	\begin{itemize}
		\item \textbf{Система билинейных уравнений} "--- это система уравнений следующего вида:
		$$
			\sum_{j = 1}^{m} \sum_{k = 1}^{n} a_{i j k} r_j s_k = d_i,~\text{для } i = 
			1,\dots,l
		$$
		\item Компактная запись параметров системы "--- $(l; m, n)$
		\item Система называется \textbf{однородной}, если $\textbf{d} = \textbf{0}$. 
	 	\item В однородных 	системах всегда имеют место \textbf{тривиальные} решения: $\textbf{s} =  
			  \textbf{0}$ или $\textbf{r} = \textbf{0}$.
		\item Система, имеющие только тривиальные решения называется \textbf{регулярной}.
		\item В данной работе рассматриваются только однородные системы.
	\end{itemize}
}
\subsection{Связь с задачей об однородности}
\frame
{
	\frametitle{Задача об однородности}
	
	В задаче об однородности аффинных поверхностей возникают 
	системы однородных билинейных уравнений особого вида : 
	$$l = m + n - 1,\, m = n.$$ 
	Особый интерес представляют нерегулярные системы.
	Примеры возникающих систем:
	\begin{itemize}
		\item Случай $(15; 8, 8)$ "--- все стандартные методы решения нелинейных систем (например, базис Грёбнера) не дают результатов. 
		\item В связи с этим имеет смысл исследовать более простой, <<модельный>> случай $(3;2,2)$, чтобы понять природу нерегулярных систем.
	\end{itemize}
}
\section{Частный случай}
\subsection{Постановка задачи}
\frame
{
	\frametitle{Случай $(3; 2, 2)$}
	Любая билинейная система является линейной по одному из наборов переменных. Для случая 	
	$(3;2, 2)$:
	$$
		\underbrace{
		\begin{pmatrix}
			a_{1 1} s_1 + a_{1 2} s_2 & a_{1 3} s_1 + a_{1 4} s_2 \\
			a_{2 1} s_1 + a_{2 2} s_2 & a_{2 3} s_1 + a_{2 4} s_2 \\
			a_{3 1} s_1 + a_{3 2} s_2 & a_{3 3} s_1 + a_{3 4} s_2 \\
		\end{pmatrix}}_{A}
		\times
		\begin{pmatrix}
			r_1 \\
			r_2
		\end{pmatrix}=
		\begin{pmatrix}
			0 \\
			0 \\
			0
		\end{pmatrix}
	$$
	Эта система имеет нетривиальные решения, если $\text{rank}\,A 
	\ne 3$, что означает, что все миноры третьего порядка в матрице $A$ равны нулю. 
}
\subsection{Достаточное условие регулярности системы}
\frame
{
	\frametitle{Достаточное условие регулярности системы}
	
}
\end{document}